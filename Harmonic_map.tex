\section{Harmonic maps}
\label{sec:preliminaries}

Given two compact manifolds $(M,g)$ and $(N,h)$ we define the smooth
map $F:M\ra N$. To comply with the terminology used in literature, we
will from now on call $M$ the domain manifold and $N$ the target
manifold. We shall construct the simplest possible scalar,
which would engage the metric tensors of both $(M,g)$ and $(N,h)$ and the mapping $F$.\\
Given the basis $e^i$ on $T_x M$, the simplest scalar function on $T_x
M\otimes T_x M$ is the scalar product $<,>_x$ defined as
\begin{align}
  \label{eq:1}
  <e^i\otimes e^j,e^k\otimes e^l>_x=g^{ik}g^{jl}.
\end{align}
For two tensors $\tau$ and $\tau'$ from $T_x M\otimes T_x M$ we then
have
\begin{align}
  \label{eq:2}
  <\tau,\tau'>_x=\tau^{ij}\tau_{ij}'.
\end{align}
Given the map $F$, we can construct the pullback $F^*h\in T_x M\otimes
T_x M$ and use the above scalar product to contract $g$ and $F^* h$
and therefore build up a scalar function we intended
\begin{align}
  \label{eq:3}
  e(F):=\frac{1}{2}<g,F^*h>_x.
\end{align}
This is the generalization of the Dirichlet energy density
$\frac{1}{2}\lVert\nabla u\rVert^2$ for functions $u:M\ra\mathbb{R}$,
indeed, for real function $u$ we have
\begin{align}
  \label{eq:4}
  e(u)=\frac{1}{2}<g,u^*1>=\frac{1}{2}g^{ij}\partial_i u\partial_j u
  =\frac{1}{2}\lVert\nabla u\rVert^2.
\end{align}
(TODO: the generalization of harmonic function)
\begin{definition}\label{def:regular-map}
  We say that a map $F$ is regular if the corresponding Dirichlet
  energy density is finite
  \begin{align}
    \label{eq:5}
    e(F)<\infty.
  \end{align}
\end{definition}

Integrating $e(F)$ over $M$, we obtain the Dirichlet energy of the
mapping $F$
\begin{align}\label{eq:En_general}
  E(F)=\frac{1}{2}\int_M e(F)dV_M.
\end{align}
% For various choices of the source manifold and the target manifold, we
% obtain various models, which by physicists are also called the sigma
% models.

By defining a functional on the space of maps $M\ra N$ we distinguish
the class of maps for which the functional is extremalized. Depending
on the domain manifold, the extrema of \eqref{eq:En_general} have
different names in literature. We will be dealing with harmonic maps
with Riemannian manifold as a domain, or with wave maps if the domain
is a Minkowski manifold.

\begin{remark}
  Dirichlet energy of any map from a Riemannian manifold to a
  Riemannian manifold is non-negative.
\end{remark}

By choosing the local coordinate frames
$x^a$ on $M$ and $F^A$ on $N$ we can write \eqref{eq:En_general} as
\begin{align}
  \label{eq:6}
  E(F)=\frac{1}{2}\int_M h_{AB}(F)\frac{\partial F^A}{\partial
    x^a}\frac{\partial F^B}{\partial x^b}g^{ab}dV_M.
\end{align}

% which resembles the Lagrangian for a sigma model (TODO:example).
By the variational calculus, the critical points of \eqref{eq:6} are
the solutions to the following Euler-Lagrangian equations
\begin{align}
  \label{eq:7}
  \Delta_g F^C+(\Gamma_{h(F)})_{AB}^{C}\frac{\partial F^A}{\partial
    x^a}\frac{\partial F^B}{\partial x^b}g^{ab}=0,
\end{align}
where $\Delta_g$ is the Laplace-Beltrami operator on $M$ and
$\Gamma_{h(F)}$ is the Christoffel symbol of the Levi-Civita
connection on $N$. From now on, we shall assume that both, the domain
and the target manifolds, are Riemannian, and in consequence the above
set of semi-linear PDE's is elliptic.\\

The problem of existence of nontrivial solutions to \eqref{eq:7} in
general is still open, but there are variety of partial results
including e.g.

\begin{theorem}[Eells-Sampson
  \cite{Eells1964}]\label{thm:Eells-Sampson}
  If $N$ is compact and has non-positive Riemannian curvature, then
  every homotopy class of maps $M\ra N$ contains a harmonic map whose
  energy is an absolute minimum in the given homotopy class.
  % Suppose $M$ is compact, $\partial M=\varnothing$ and that the
  % sectional curvature $K^N$ of $N$ is non-positive. Then there exists
  % a harmonic map of any given homotopy class.
\end{theorem}

One can also easily verify that the identity map $\id:M\ra M$ is a
harmonic map, regardless of the choice of $M$, by substituting
$F^A=x^A$ into \eqref{eq:7}. The energy density of the identity map
is $e(\id)=\dim(M)/2$.\\

\begin{remark}\label{rem:1}
  The identity map $\text{\emph{id}}:M\ra M$ is harmonic.
\end{remark}

The following remark can be also proved easily.

\begin{remark}\label{rem:2}
  Given two harmonic maps $F_1:M_1\ra N_1$ and $F_2:M_2\ra N_2$, the
  map $F:M_1\times M_2\ra N_1\times N_2$ of the form
  $F((x_1,x_2))=(F_1(x_1),F_2(x_2))$ is also harmonic.
\end{remark}

After this very brief introduction to harmonic maps we proceed to the
problem to which this thesis is devoted.


%%% Local Variables:
%%% mode: latex
%%% TeX-master: "master"
%%% End:

\section{Preliminaries}
\label{sec:preliminaries}


The energy functional for map $f:M\ra N$ between Riemannian manifolds
$(M,g)$ and $(N,h)$ is defined by
\begin{align}\label{eq:En_general}
  E(u)=\frac{1}{2}\int_M h_{AB}(f)\frac{\partial f^A}{\partial
    x^i}\frac{\partial f^B}{\partial x^j}g^{ij}dV_M\ge0
\end{align}
in local coordinates $x^i$ in $M$ and $f^A$ in $N$. In our case
$M=N=S^k$ with $k\ge3$. Now we use the setup from [Bizon] (TODO excuse
such setup and the details of the domain of the functional), which
simplifies \eqref{eq:En_general} to
\begin{align}
  \label{eq:En_Sk}
  E(u)=\frac{1}{2} V(S^{k-1})\int_{-\pi/2}^{\pi/2}
  \left(f'^2+(k-1)\frac{\sin^2f}{\sin^2\psi}\right) \sin^{k-1}\psi
  d\psi.
\end{align}
The term $V(S^{k-1})$ has no qualitative impact on the behaviour of
the system and we shall drop it from now on. Critical points of such
functional are solutions to the corresponding Euler-Lagrange equation
\begin{align}
  \label{eq:f_psi_EL}
  -\frac{\delta E}{\delta f}=\frac{1}{\sin^{k-1}\psi}\left(\sin^{k-1}\psi
    f'\right)'-\frac{(k-1)}{2}\frac{\sin2f}{\sin^2\psi}=0.
\end{align}

This equation has been extensively studied by Bison and Wald and the
solutions are well known with their properties described in
[]. However, we will approach this problem with the means of the
gradient flow by considering the flow
\begin{equation}
  \label{eq:en_flow}
  \begin{split}
    f_t&=-\frac{\delta E}{\delta f}=\frac{1}{\sin^{k-1}\psi}\left(\sin^{k-1}\psi
      f'\right)'-\frac{(k-1)}{2}\frac{\sin2f}{\sin^2\psi}\\
    f\left(t,0\right)&=0\\
    f\left(t,\pi\right)&=\delta\pi\\
    \delta\in{0,1}
  \end{split}
\end{equation}
Such flow is also known as harmonic map flow because stationary points
of it coincide with the harmonic maps as they are solutions of the ODE
\eqref{eq:f_psi_EL}. The choice of $\delta$ reduces the class of
possible stationary points to maps of homotopy degree up to one. Under
evolution according to \eqref{eq:en_flow} the energy decreases
\begin{align}
  \label{eq:en_t}
  E_t(f)&=\lim_{h\ra0}\frac{E(f(t+h))-E(f(t))}{h}\\
  &=\lim_{h\ra0}\frac{E(f+h f_t+\mathcal{O}(h^2))-E(f)}{h}\\
  % =\lim_{h\ra0}\left(\frac{\delta E}{\delta f}(f_t)+\mathcal{O}h\right)\\
  &=\int_{-\pi/2}^{\pi/2}\frac{\delta E}{\delta f}f_t\sin^{k-1}\psi d\psi\\
  &=-\int_{-\pi/2}^{\pi/2}f_t^2\sin^{k-1}\psi d\psi\le0
\end{align}
with $E_t(f)=0$ iff $f_t=0$. On the other hand, $E(f)\ge0$ so for
generic initial data we would expect energy to asymptotically decrease
to one of its local minimas. Such minimas can be located only at
stationary points which are not unstable under linear
perturbation. Due to the existence of conformal Killing field
$K=\sin(\psi)\frac{\partial}{\partial \psi}$ (with $\mathcal{L}_K
g=-2\cos\psi g$ TODO:check) the only local minimas are constant
solutions $f_0=k\pi$. One can show this by analysing the second
variation of the energy along the $v:=\mathcal{L}_K f=\sin\psi
f'(\psi)$ around the critical point
\begin{align}
  \delta^2E(v,v)
  &=\int_{-\pi/2}^{\pi/2}
  \left(
    v'^2+(k-1)\frac{\cos2f}{\sin^2\psi}v^2
  \right)\sin^{k-1}\psi d\psi\\
  &=-\int_{-\pi/2}^{\pi/2}
  \left(\frac{1}{\sin^{k-1}\psi}\left(\sin^{k-1}\psi
      v'\right)'-(k-1)\frac{\cos2f}{\sin^2\psi}v\right)v\sin^{k-1}\psi
  d\psi\\
  &=-\int_{-\pi/2}^{\pi/2}(Lv)(\psi)v(\psi)\sin^{k-1}\psi d\psi
\end{align}
Now from \eqref{eq:strange_variation} for critical points we have
$Lv=(k-2)v$ and thus $\delta^2 E(v,v)\le0$ and the direction
$v=\sin\psi f'(\psi)$ is unstable. This argument works for any
nonconstant $f$ which is the solution to \eqref{eq:f_psi_EL}. For any
constant solution such vector is null and is the trivial solution to
the linearized equation. On the other hand the only constant solutions
to \eqref{eq:f_psi_EL} are $f_0=0$, $\bar{f}_0=\pi$ and $f_e=\pi/2$
($e$ as in \emph{equatorial}) and using the results from the following
sections we claim that $f_0$ and $\bar{f}_0$ are linearly stable but
$f_e$ is of infinite index for $3\le k\le6$, thus for such choice of
$k$ it is the local maximum rather then local minimum.

% Now when we have briefly introduced the possible local attractors we
% shall ask what is their impact on the global dynamics.
% Apart from being linearly stable, $f_0$ and $\bar{f}_0$ are separated
% (TODO: in what sense?) global minimas of energy functional
% ($E(f_0)=E(\bar{f}_0)=0$), so we would expect that for generic initial
% data the solution should fall into one of such attractors. However, we
% shall see that in the phase space there exists the set of solutions
% that fall to neither of the attractors and thus imply the existence of
% nontrivial solution to \eqref{eq:f_psi_EL}.

(TODO: condition for solution to converge to particular constant
solution)

%%% Local Variables:
%%% mode: latex
%%% TeX-master: "master"
%%% End:

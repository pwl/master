\section{Harmonic maps for $F:S^k\ra S^k$}
\label{sec:harmonic-maps-skra}

From now on we shall set both, the domain and target manifolds to
$S^k$. We will use the equation \eqref{eq:31} in coordinates

\subsection{Basic setup}
\label{sec:basic-setup}

We choose the coordinate frame on the domain sphere as
$(\psi,\theta)$, where $\psi\in(0,\pi)$ is the longitudal angle with
south pole at $\psi=0$ and $\theta$ is the set of coordinates on
$S^{k-1}$ -- the equator of $S^k$. Analogously we introduce
coordinates $(\Psi,\Theta)$ on the target sphere. The metric tensors
for the given coordinate frames are
\begin{align}
  \label{eq:23}
  &\text{domain:}&\quad &ds^2=d\psi^2+\sin^2\psi^2ds^2\big|_{S^{k-1}}\\
  &\text{target:}&\quad &dS^2=d\Psi^2+\sin^2\Psi^2dS^2\big|_{S^{k-1}}.
\end{align}
Solving equations \eqref{eq:31} without any further assumptions
presents an impossible task so we shall simplify our problem by
assuming that $\Theta=\theta$ and $\Psi=f(\psi)$ so the map $F$ takes
the form
\begin{align}
  \label{eq:24}
  F:(\psi,\theta)\ra(f(\psi),\theta),
\end{align}
which leaves us with one function as a degree of freedom. The given
setup has been introduced in ~\cite{Eells1964} and it consists of the
idea that after removing the poles $S^k$ can be regarded as
$(0,\pi)\times S^{k-1}$, and by applying remarks \ref{rem:1} and
\ref{rem:2}.\\

For $F$ to be continuous, we require that
\begin{align}
  \label{eq:27}
  \lim_{\psi\ra0}f(\psi)=0\quad\lim_{\psi\ra\pi}f(\psi)=m\pi.
\end{align}
The choice of $f\ra0$ for $\psi\ra0$ was only the matter of
convenience -- we haven't lost any generality because the situation is
symmetric
w.r.t. $f\ra f+n\pi$. Moreover, closing the domain of $\psi$ will not
have any implications so we can drop the limits from \eqref{eq:27} \\

The Dirichlet energy of the considered map now has a more transparent
form
\begin{align}
  \label{eq:En_Sk}
  E(f)=\frac{1}{2} V(S^{k-1})\int_{0}^{\pi}
  \left(f'^2+(k-1)\frac{\sin^2f}{\sin^2\psi}\right) \sin^{k-1}\psi
  d\psi.
\end{align}
Where we have changed the argument of $E$ from $F$ to $f$ as
effectively it is a functional of $f$. The term $V(S^{k-1})$ has no
qualitative impact on the behaviour of the system and we shall drop it
from now on. The notable fact is the nontrivial $Z_2$ symmetry of the
functional $E(f)$ manifested as the reflection
\begin{align}
  \label{eq:33}
  f&\ra\pi-f,\\
  E(f)&\ra E(\pi-f)=E(f),
\end{align}
leads to the immediate conclusion that every possible critical point
will have its partner of the same homotopy degree (they will however
not fulfill our artificial boundary condition $f(0)=0$, but we leave
this problem until the next sections). This does not hold for the
function invariant to \eqref{eq:33}, denoted as $f_e=\frac{\pi}{2}$,
which by \eqref{eq:27} is not continuous, so it
cannot be harmonic.\\

Critical points of $E(f)$ are solutions to the corresponding
Euler-Lagrange equation
\begin{align}
  \label{eq:f_psi_EL}
  \frac{1}{\sin^{k-1}\psi}\left(\sin^{k-1}\psi f'\right)'-\frac{(k-1)}{2}\frac{\sin2f}{\sin^2\psi}=0.
\end{align}
with the boundary values
\begin{align}
  \label{eq:32}
  f(0)=0\quad f(\pi)=m\pi.
\end{align}
The boundary conditions imply that the acceptable solutions of
\eqref{eq:f_psi_EL} behave at $\psi=0$ as $f\sim\psi$, and analogously
for $\psi=\pi$, $f\sim(\psi-\pi)$. Moreover, \eqref{eq:f_psi_EL} has
trivial solution of the form $f=0$. By remark \ref{rem:1}, the
identity map $f=\psi$ is also the solution. This is all what can we
say about \eqref{eq:f_psi_EL} without diving into the theory of
ODE's. The following section presents merely the results of the works
~\cite{Corlette2001,Bizon1997}, where the infinite countable family of
solutions to \eqref{eq:f_psi_EL} have been proved to exist. TO BE CONTINUED\\


The possible solutions of the above equation have been extensively
studied in ~\cite{Corlette2001,Bizon1997}, where the two infinite
countable family of solutions have been proved to exist. The
representatives of each family are harmonic maps of degree zero or one
only. The authors have also shown a few properties of the possible
solutions which are summarized in the following paragraph.

\subsection{Solution family}
\label{sec:solution-family}

We shall denote a solution to \eqref{eq:f_psi_EL} as $f_n$,
$n\in\mathbb{N}_0$ with $f_n$ being the map of degree zero for $n$
even, and of degree one for $n$ odd. Each of $f_n$ has a partner
$\bar{f}_n(\psi)=\pi-f_n(\psi)$, as a consequence of the
$\mathbb{Z}_2$ symmetry of \eqref{eq:f_psi_EL}, which is also a
solution of the same degree. The first two solutions of degree zero,
are the two trivial ones, $f_0=0$ and $\bar{f}_0=\pi$, which are the
only regular constant solutions. Next, corresponding to $n=1$, are the
identity and anti-identity maps $f_1=\psi$ and $\bar{f}_1=\pi-\psi$,
which are representatives of the degree one family. The solutions for
$n>1$ have no closed form, but from ~\cite{Bizon1997} we know that
each $f_n$ has exactly $n$ intersections with $\pi/2$ and $n-1$
extrema (the latter holds for $n\ge1$).\\

Due to the existence of conformal Killing field
$K=\sin(\psi)\frac{\partial}{\partial \psi}$ (with $\mathcal{L}_K
g=-2\cos\psi g$ TODO:check) the only local minimas are constant
solutions $f_0=k\pi$. One can show this by analysing the second
variation of the energy along the $v:=\mathcal{L}_K f=\sin\psi
f'(\psi)$ around the critical point
\begin{align}
  \delta^2E(v,v)
  &=\int_{-\pi/2}^{\pi/2}
  \left(
    v'^2+(k-1)\frac{\cos2f}{\sin^2\psi}v^2
  \right)\sin^{k-1}\psi d\psi\\
  &=-\int_{-\pi/2}^{\pi/2}
  \left(\frac{1}{\sin^{k-1}\psi}\left(\sin^{k-1}\psi
      v'\right)'-(k-1)\frac{\cos2f}{\sin^2\psi}v\right)v\sin^{k-1}\psi
  d\psi\\
  &=-\int_{-\pi/2}^{\pi/2}(Lv)(\psi)v(\psi)\sin^{k-1}\psi d\psi
\end{align}
Now from \eqref{eq:strange_variation} for critical points we have
$Lv=(k-2)v$ and thus $\delta^2 E(v,v)\le0$ and the direction
$v=\sin\psi f'(\psi)$ is unstable. This argument works for any
nonconstant $f$ which is the solution to \eqref{eq:f_psi_EL}. For any
constant solution such vector is null and is the trivial solution to
the linearized equation. On the other hand the only constant solutions
to \eqref{eq:f_psi_EL} are $f_0=0$, $\bar{f}_0=\pi$ and $f_e=\pi/2$
($e$ as in \emph{equatorial}) and using the results from the following
sections we claim that $f_0$ and $\bar{f}_0$ are linearly stable but
$f_e$ is of infinite index for $3\le k\le6$, thus for such choice of
$k$ it is the local maximum rather then local minimum.

% Now when we have briefly introduced the possible local attractors we
% shall ask what is their impact on the global dynamics.
% Apart from being linearly stable, $f_0$ and $\bar{f}_0$ are separated
% (TODO: in what sense?) global minimas of energy functional
% ($E(f_0)=E(\bar{f}_0)=0$), so we would expect that for generic initial
% data the solution should fall into one of such attractors. However, we
% shall see that in the phase space there exists the set of solutions
% that fall to neither of the attractors and thus imply the existence of
% nontrivial solution to \eqref{eq:f_psi_EL}.

(TODO: condition for solution to converge to particular constant
solution)


%%% Local Variables:
%%% mode: latex
%%% TeX-master: "master"
%%% End:

\section{Harmonic maps for $F:S^k\ra S^k$}
\label{sec:harmonic-maps-skra}

\subsection{General properties}
\label{sec:general-properties}

From now on we shall set both, the domain and target manifolds to
$S^k$. We choose the coordinate frame on the domain sphere as
\begin{align}
  \label{eq:46}
  x^a=(\psi,\theta),
\end{align}
where $\psi\in(0,\pi)$ is the longitudal angle with
south pole at $\psi=0$ and $\theta$ is the set of coordinates on
$S^{k-1}$ -- the equator of $S^k$. Analogously we introduce
coordinates $(\Psi,\Theta)$ on the target sphere in which the map $F$
takes the form
\begin{align}
  \label{eq:47}
  F^A(\psi,\theta)&=(\Psi,\Theta).
\end{align}
The metric tensors for the given coordinate frames are
\begin{align}
  \label{eq:23}
  &\text{domain:}&\quad &ds^2=d\psi^2+\sin^2\psi^2ds^2\big|_{S^{k-1}}\\
  &\text{target:}&\quad &dS^2=d\Psi^2+\sin^2\Psi^2dS^2\big|_{S^{k-1}}.
\end{align}
Solving equations \eqref{eq:31} without any further assumptions
presents an impossible task, we therefore in section
\ref{sec:basic-setup} introduce a simple ansatz. Still without any
simplifications we can state the following.

\begin{theorem}\label{thm:skk-energy-bound}
  For $k\ge3$ and any given map $F:S^k\ra S^k$ there is a map within
  the homotopy class of $F$ of arbitrary small Dirichlet energy.
\end{theorem}

\begin{proof}
  The proof bases on the fact that on any sphere there exists a one
  parameter group of conformal maps, which in the coordinates
  \eqref{eq:46} has the form
  \begin{align}
    \label{eq:10}
    \psi_A=2\arctan(e^A\tan(\psi/2)).
  \end{align}
  The above conformal map can be depicted as dragging the whole sphere
  along the longitudal coordinates in the direction of one of its poles
  (for $A$ large, this would be the north pole). We define the map $F_A$
  to be a composition
  \begin{align}
    \label{eq:11}
    F_A=F\circ\psi_A.
  \end{align}
  Due to the conformal properties of the map $\psi_A$ we obtain the
  following energy density of $F_A$
  \begin{align}
    \label{eq:12}
    e(F_A)_\psi=&e(F)_{\psi_A}\rho_A^2,\\
    \rho_A=&\frac{1}{\cosh A-\cos\psi\sinh A},
  \end{align}
  where $e(F)_\psi$ is the energy density of $F$ at point $\psi$.  The
  map $F$ is regular, and therefore $e(F)_{\psi_A}$ is bounded by its
  maximal value $C(F)=\max_\psi\left(e(F)_\psi\right)$ and we have the
  following bound
  \begin{align}
    \label{eq:13}
    e(F_A)_\psi&\le C(F)\rho_A^2.
  \end{align}
  Assuming $k\ge3$, the Dirichlet energy of $F_A$ can be bounded by
  \begin{align}
    \label{eq:14}
    E(F_A)&=\int_{S^k}e(F_A)dV_{S^k}\\
    &\le C(F)V(S^{k-1})\int_{0}^{\pi}\rho_A^2\sin^{k-1}\psi d\psi\\
    &\le C(F)V(S^{k-1})\max_\psi(\sin^{k-3}\psi)\int_{0}^{\pi}\rho_A^2\sin^2\psi d\psi\\
    &\le C_1(F,k)\frac{1}{1+\cosh A}.
  \end{align}
  which can be made arbitrary small by an appropriate choice of $A$. The
  above bound is not optimal, but is enough for our purposes. As the map
  $\psi_A$ leaves the north and south poles of $S^k$ unchanged, the map
  $F_A$ is of the homotopy class that $F$ and we conclude by the
  following theorem.

\end{proof}



\subsection{Harmonic map ansatz}
\label{sec:basic-setup}

We simplify our problem by assuming that $\Theta=\theta$ and
$\Psi=f(\psi)$ so the map $F$ takes the form
\begin{align}
  \label{eq:24}
  F:(\psi,\theta)\ra(f(\psi),\theta),
\end{align}
which leaves us with one function as a degree of freedom. The given
setup has been introduced in ~\cite{Eells1964} and it consists of the
idea that, after removing the poles, $S^k$ can be treated as
$(0,\pi)\times S^{k-1}$, for which we use remarks \ref{rem:1} and
\ref{rem:2}.\\

For $F$ to be continuous, we require that
\begin{align}
  \label{eq:27}
  \lim_{\psi\ra0}f(\psi)=n\pi\quad\lim_{\psi\ra\pi}f(\psi)=m\pi.
\end{align}
Moreover, closing the domain of $\psi$ will not have any implications
as long as $F$ is regular so we can drop the limits from \eqref{eq:27}
\begin{align}
  \label{eq:1}
  f(0)=n\pi\quad f(\pi)=m\pi.
\end{align}
The number $\lvert n-m\rvert$ stands for the homotopy degree of a
map.\\

The Dirichlet energy of the considered map now has a more transparent
form
\begin{align}
  \label{eq:En_Sk}
  E(f)=\frac{1}{2} V(S^{k-1})\int_{0}^{\pi}
  \left(f'^2+(k-1)\frac{\sin^2f}{\sin^2\psi}\right) \sin^{k-1}\psi
  d\psi.
\end{align}
Where we have changed the argument of $E$ from $F$ to $f$ as
effectively it is a functional of $f$. The term $V(S^{k-1})$ has no
qualitative impact on the behaviour of the system and we shall drop it
from now on. The energy density of \eqref{eq:En_Sk} with the volume
term $V(S^{k-1})$ dropped is
\begin{align}
  \label{eq:35}
  e(f)=\frac{1}{2}\left(f'^2+(k-1)\frac{\sin^2f}{\sin^2\psi}\right).
\end{align}
By the definition \ref{def:regular-map}, the map $f$ is regular if
\begin{align}
  \label{eq:38}
  e(f)=\frac{1}{2}\left(f'^2+(k-1)\frac{\sin^2f}{\sin^2\psi}\right)<\infty.
\end{align}
The Dirichlet energy of $f$ is
\eqref{eq:En_Sk}
\begin{align}
  \label{eq:36}
  E(f)=\int_0^\pi e(f)\sin^{k-1}\psi d\psi\ge0.
\end{align}
The notable fact is the nontrivial $Z_2$ symmetry of the
functional $E(f)$ manifested as the reflection
\begin{align}
  \label{eq:33}
  f&\ra\pi-f,\\
  E(f)&\ra E(\pi-f)=E(f),
\end{align}
leads to the immediate conclusion that every possible critical point
will have its partner of the same homotopy degree% (they will however
% not fulfill our artificial boundary condition $f(0)=0$, but we leave
% this problem until the next sections)
. This does not hold only for the function invariant to
\eqref{eq:33}. The only function such that $f=\pi-f$ is $f_e=\pi/2$,
which corresponds to mapping the whole sphere to an equatorial (hence
index $e$), but by \eqref{eq:27} it is not continuous, so it cannot be
harmonic. The map $f_e$ will however play an important role
later on, and it is analysed in more detail in the appendix.\\

Critical points of $E(f)$ are solutions to the corresponding
Euler-Lagrange equation
\begin{align}
  \label{eq:f_psi_EL}
  \frac{1}{\sin^{k-1}\psi}\left(\sin^{k-1}\psi f'\right)'-\frac{(k-1)}{2}\frac{\sin2f}{\sin^2\psi}=0.
\end{align}
with the boundary values
\begin{align}
  \label{eq:32}
  f(0)=n\pi\quad f(\pi)=m\pi.
\end{align}
The equation \eqref{eq:f_psi_EL} has trivial solutions of homotopy
degree zero and of the form $f=n\pi$ which attains the global energy
minimum. By remark \ref{rem:1}, the identity map $f=\psi+n\pi$ and, by
\eqref{eq:33}, its image $\bar{f}=-\psi+n\pi$ are also solutions but
of the homotopy degree one. These are the only solutions to
\eqref{eq:f_psi_EL} known in the closed form. Because in general, the
symmetry
\begin{align}
  \label{eq:2}
  E(f+n\pi)=E(f)
\end{align}
implies that any stationary point of $E$ can remain stationary after
being translated by $n\pi$ we henceforth drop the inessential
``$+n\pi$'' to simplify the notion of possible solutions.

The solutions to \eqref{eq:f_psi_EL} are analysed in a detailed way in
papers \cite{Corlette2001,Bizon1997}, where for $3\le k\le6$ the
infinite countable family of solutions have been proved to exist, and
some of their important properties have been revealed.\\

\subsection{Family of nontrivial harmonic maps}
\label{sec:solution-family}

The authors of \cite{Bizon1997} have found out that any regular
solution to \eqref{eq:f_psi_EL} will stay inside the box
$[0,\pi]\times[0,\pi]$, which implies that there are no solutions of
homotopy degree higher than one. For $k\ge7$ no solutions apart from
constant and identity ones exist, but for $3\le k\le 7$ there is a
family of solutions, parametrized by their nodal number being the
number of intersections with $\pi/2$. To be consistent with
\cite{Bizon1997} we denote a solution to \eqref{eq:f_psi_EL} as $f_n$,
$n\in\mathbb{N}_0$ with $f_n$ being the map of degree zero for $n$
even, and of degree one for $n$ odd.

The first solution of degree zero is the trivial one, $f_0=0$ and
which is the only constant regular one. Next, corresponding to $n=1$,
is the identity map $f_1=\psi$, which is the representative of the
degree one family. The solutions for $n>1$ have no closed form, but
from \cite{Bizon1997} we know that each $f_n$ has exactly $n$
intersections with $\pi/2$ and $n-1$ extrema (the latter holds for
$n\ge1$). The functions $f_n$ converge in a non-uniform way to the
solution $f_e$ which due to the convention is also denoted as
$f_\infty$. In \cite{Corlette2001} the authors prove that each $f_n$
is of Morse index $n$, thus $f_0$ is the only local minimum of
$E(f)$. (TODO: A few of the first solutions are depicted in the
figure).


%%% Local Variables:
%%% mode: latex
%%% TeX-master: "master"
%%% End:

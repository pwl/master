\subsection{The case $k\in\Gamma_1$}
\label{sec:case-kingamma_1}

As $K$ is already self-adjoint we shall proceed to solving the
eigenproblem
\begin{align}
  \label{eq:21}
  -f''(\psi)+\frac{1}{4}(k-7)(k-1)\cot^2\psi f(\psi)=\lambda f(\psi)
\end{align}
The general solutions to above ODE are the conical functions
\begin{align}
  \label{eq:1}
  f_\lambda&=\sqrt{\sin\psi}
  \left(AP_{\beta}^{\alpha}(\cos\psi)+BP_{\beta}^{-\alpha}(\cos\psi)\right)\\
  \alpha&=\frac{1}{2}\sqrt{k^2-8k+8}>0\\
  \beta&=\frac{1}{2}\left(-1+\sqrt{4\alpha^2-1+4\lambda}\right).
\end{align}
(TODO: $\Lambda_m>0$) As $\sqrt{\sin\psi}P_\beta^{\alpha}(\cos\psi)\sim\psi^{1/2-\alpha}$
grows too rapidly at $\psi=0$, we set $A=0$. We now analyse the
behaviour at $\psi=\pi$ by expanding $P_\beta^{-\alpha}$ around
$\cos\pi=-1$
\begin{align}
  \label{eq:2}
  &P_\beta^{-\alpha}(z)=\\
  &\frac{2^{\alpha/2}\Gamma(\alpha)}{\Gamma(\alpha-\beta)\Gamma(\beta+\alpha+1)}
  (z+1)^{-\alpha/2}(1+\mathcal{O}((z+1)))\\
  &-\frac{1}{\pi}2^{-\alpha/2}\sin(\pi\beta)\Gamma(-\alpha)
  (1+z)^{\alpha/2}(1+\mathcal{O}((z+1)))
\end{align}
We obtain the quantization of $\lambda$ by setting the coefficient in
the first term to zero. This can be achieved only by setting $\beta$
so that one of the functions $1/\Gamma(z)$ will zero, which induces
$\beta\in\R$. But then $\beta+\alpha+1>0$ and the only possible zeroes
are for
\begin{align}
  \label{eq:3}
  \alpha-\beta=-m\\
  m\in\mathbb{N}_0
\end{align}
which gives rise to the following quantization of $\Lambda$
\begin{align}
  \label{eq:6}
  \Lambda_m&=m^2+m(1+2\alpha)+\alpha+2-\frac{3}{2}k\\
  &=(m+\frac{1}{2}(1+2\alpha))^2-\frac{1}{4}(k-1)^2\\
  &=(m+\alpha+1-\frac{k}{2})(m+\alpha+\frac{k}{2})\\
  &>0\quad\text{for } k\in\Gamma_1 (TODO, not true!!)
\end{align}
so, the equatorial map is stable for $k\in\Gamma_1$.  The general
solutions to \eqref{eq:21} form an orthogonal basis
\begin{align}
  \label{eq:11}
  f_m(\psi)=\sqrt{\sin\psi}P_{\alpha+m}^{-\alpha}(\cos\psi)
\end{align}
The sum of the coefficients $\alpha$ and $\alpha-m$ is an non-positive
integer $-m$ which means that $P_{\alpha+m}^{-\alpha}(z)$ and
$P_{\alpha+m}^{-\alpha}(-z)$ are linearly dependent, therefore
$f_m(\psi)$ is either, symmetric or antisymmetric
w.r.t. $\psi\ra\pi-\psi$.


%%% Local Variables:
%%% mode: latex
%%% TeX-master: "master"
%%% End:

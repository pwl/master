\subsection{The case $k\in\Gamma_2$}
\label{sec:case-kingamma_2}

Under such condition for $k$, the domain of $K^*$ is the set of
functions from $\LSq$ and both solutions to \eqref{eq:14} are
valid. However, one of the solutions blows-up at $\psi=0$ as
$\psi^{1/2-\alpha}$ while the other one tends to zero as
$\psi^{1/2+\alpha}$, the same happens at $\psi=\pi$. It seems natural,
to select the self-adjoint extension which would allow only the
eigenvectors which fall off to zero at $\psi=0$ and $\psi=\pi$.
To achieve this we set the extended domain of $K$ to be
\begin{align}
  \label{eq:13}
  \D{K_2}=\{f\in C^\infty(0,\pi)|\quad f(0)=f(\pi)=0\}.
\end{align}
With respect to the above domain, the domain of the operator and its
adjoint have to be the same and thus $K_2=K_2^*$. It can be
seen by utilizing the fact, that for every vector $g\in\D(K_2^*)$
and $f\in\D(K_2^*)$ we shall have
\begin{align}
  \label{eq:17}
  0&=(g,K_2 f)-(K_2^*g,f)\\
  &=f'(\pi)g(\pi)-f(\pi)g'(\pi)+f'(0)g(0)-f(0)g'(0)\\
  &=f'(\pi)g(\pi)-f'(0)g(0).
\end{align}
This implies $g(0)=g(\pi)=0$. Thus $K_2$ is one of the possible
self-adjoint extensions of $K$. As $\alpha$ is real for $k\in\Gamma_2$
the same reasoning as in \ref{sec:case-kingamma_1} holds and we find
the eigenvectors of $K_2$ to be the same which is another argument
behind choosing such extension. Such selection of the domain however,
produces self-adjoint extension only for $\alpha>1/2$ for which one of
the solutions blows up at $0$ and for the special case $k=7$ and $k=1$
for which $\alpha=1/2$ and one of the solutions is constant near the
boundaries.

As a matter of fact if we had chosen the domain \eqref{eq:13} in the
beginning we would have not reached the necessity of extending $K$ for
$\Gamma_2$. However, we have seen that such choice of the extension is
not an unique one thus without continuity in eigenvalues w.r.t $k$ as
a motivation we would not have any justification in our choice of
$K_2$. Another motivation is that the presence of any admixture of the
modes with singular behaviour at the boundaries in the spectrum would
produce, during the dynamical evolution, the singular behaviour even
for initial data without a singularities at the boundaries.


% To
% achieve this we have to calculate more explicitly how the choice of
% the possible extension of $K$, lets say $T$,
% \begin{align}
%   \label{eq:4}
%   T=-\frac{d^2}{d\psi^2}+V(\psi)
% \end{align}
% influences the domain of its adjoint. We have introduced the new
% letter $T$\\

% For vectors $f\in\D{T}$ and $g\in\D{T^*}$ we have
% \begin{align}
%   \label{eq:1}
%   (g,Tf)-(T^*g,f)=W(f,g)(\pi)-W(f,g)(0)=0.
% \end{align}
% Where $W(f,g)$ is the Wrońskian determinant for $f$ and $g$:
% \begin{align}
%   \label{eq:3}
%   W(f,g)(\psi)=f'(\psi)g(\psi)-f(\psi)g'(\psi)
% \end{align}
% We are considering the functions which possibly blow-up at $\psi=0$ or
% $\psi=\pi$ so we have to put limits into the formula
% \begin{align}
%   \label{eq:2}
%   (g,Tf)-(T^*g,f)=\lim_{\psi\ra\pi}W(f,g)(\psi)-\lim_{\psi\ra0}W(f,g)(\psi).
% \end{align}
% Assume, that $f\in\D{T}$ behaves as $\psi^{1/2+\delta_1}$ near
% $\psi=0$, and analogously $\D{T^*}\ni g\sim\psi^{1/2+\delta_2}$ for some
% real $\delta_1$ and $\delta_2$. Then
% \begin{align}
%   \label{eq:6}
%   W(f,g)(\psi)\sim(\delta_1-\delta_2)\psi^{\delta_1+\delta_2}.
% \end{align}
% The requirement for the Wrońskian to vanish in the limit $\psi\ra0$ is
% \begin{align}
%   \label{eq:7}
%   \delta_1+\delta_2>0\quad\text{or}\quad\delta_1=\delta_2.
% \end{align}
% We are interested in the case, where the singular eigenfunctions are
% not allowed to show up in the domain of $T$, so we set
% $\delta_1=\alpha>0$ and the second condition in \eqref{eq:7} is
% included in the first one. Thus the domain of

% The similar formula can be derived for


% This can be done, by expanding the
% domain of $K$ to the set of functions which do not grow faster than
% $\psi^{1/2+\alpha}$ at $\psi=0$ and $(\pi-\psi)^{1/2+\alpha}$ at
% $\psi=\pi$.
% \begin{align}
%   \label{eq:16}
%   D_1=\{f\in C^\infty(0,\pi)|\quad
%   \forall{\epsilon>0}\exists{M>0}:|f(\psi)|+|f(-\psi)|\}.TODO
% \end{align}
% With such domain, we expect

% The fact that both solutions belong to the $\LSq$

% In this case, the imposing self-adjoint extension
% is one for which we admit only solutions behaving nicely at $\psi=0$
% and at $\psi=\pi$, thus the set
% \begin{align}
%   \label{eq:16}
%   \{f\in C^\infty(0,\pi)|\quad f(0)=f(\pi)=0\}.
% \end{align}

% This choice of the domain guarantees that the solutions

% , as the deficiency indices pair is
% $<2,2>$, we end up with the possible self-adjoint extensions of $K$
% parametrized by $U(2)$.


% However, we can simplify the situation by admitting only
% symmetric or antisymemtric perturbations equivalent to considering the
% operators $L_+$ and $L_-$ for which respective domains are
% $\D(L_+)=\{f\in C^\infty(0,\pi)|\quad f(\psi)=f(-\psi)\}$ and
% $\D(L_-)=\{f\in C^\infty(0,\pi/2)|\quad f(\psi)=-f(-\psi)\}$. $L_+$ and
% $L_-$ when considered



% a bit we shall impose some arbitrary conditions on the
% domain of $K^*$, by restricting it to functions which are symmetric or
% antisymmetric. Such restricition is motivated by the following
% chapters, where we focus on symmetric or antisymmetric solutions to
% \eqref{eq:en_flow}. Let us introduce $K_+$ and $K_-$ each acting on
% either symmetric or antisymmetric subspaces of $C_0^\infty(0,\pi)$.

% By the similar reasoning as in (Thm1) we conclude, that each, $K_1$
% and $a$

%%% Local Variables:
%%% mode: latex
%%% TeX-master: "master"
%%% End:

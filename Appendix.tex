\clearpage
\chapter{Useful identity}
\label{cha:Identity}

Given the E-L equations
\begin{align}
  \frac{1}{w}\left(wf'\right)'+V(f,x)=0
\end{align}
the second variation in the direction $v=gf'$, for $g$ arbitrary, can
be written as
\begin{align}\label{eq:strange_variation}
  \frac{1}{w}\left(wv'\right)+\frac{\partial V}{\partial
    f}v=\frac{1}{g}\left[\left(\left(\frac{g}{w}\right)'w
  \right)'v-\frac{\partial}{\partial x}\left(g^2 V(f,x)\right)\right]=A(x)v+B(f,x).
\end{align}
Where we have differentiated the E-L equation, multiplied it by $g$
and used the fact that
\begin{align}
  \frac{1}{w}\left(w(gf')'\right)'-g\left(\frac{1}{w}(gf')'\right)'
  =\left(\left(\frac{g}{w}\right)'w \right)'f'-2g'V(f,x).
\end{align}
% In all cases considered in this thesis we have $A(x)=const$ and
% $B(f,x)=0$.
% This cannot be a coincidence and such situation has to
% reflect some particular symmetry of the E-L equation. Worth noticing
% is the fact that in all cases $wg$ is eigenfunction of the laplacian

% \begin{align}
%   \frac{1}{w}\left(w(wg)'\right)'=\lambda wg
% \end{align}

% to some particular $\lambda$ which is also the eigenvalue of the first
% perturbation operator. Moreover $B=0$ implies that
% $V(f,x)=U(f)/g^2(x)$.


%%% Local Variables:
%%% mode: latex
%%% TeX-master: "master"
%%% End:

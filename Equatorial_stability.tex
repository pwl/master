\section*{Stability of equatorial map}

% For $f$ to extremalize the energy it is required that
% \begin{gather}\label{eq:el}
%   \frac{1}{\sin^{k-1}\psi}\big(\sin^{k-1}\psi
%   f'\big)'-\frac{k-1}{2}\frac{\sin 2 f}{\sin^2\psi}=0.
% \end{gather}

% TEST:
% By applying the standard linearization procedure around $f_e=\pi/2$
% we obtain the eigenproblem
% \begin{align}
%   \label{eq:problem}
%   (L-\lambda)h=h''+(k-1)\cot(\psi)h'+\left(\frac{k-1}{\sin^2\psi}-\lambda\right)h=0.
% \end{align}
% The substitution $y=\sin^2\psi$, $g(y)=h(\arcsin(y^{1/2}))$ reveals
% the Fuchsian type equation with three regular singular points at $0$,
% $1$ and $\infty$:
% \begin{gather}
%   4y(y-1)g''+2(y+k(y-1))g'+(\lambda-\frac{k-1}{y})g=0.
% \end{gather}
% The general solution of the latter is of the form
% \begin{gather}
%   g=c_1\phi_1+c_2\phi_2\\
%   \phi_1(y)=y^{\frac{-4\alpha-k+2}{4}}{}_2F_1 \left(-\alpha-\beta+\frac{1}{4},-\alpha+\beta+\frac{1}{4};1-2\alpha;y\right)\\
%   \phi_2(y)=y^{\frac{+4\alpha-k+2}{4}}{}_2F_1 \left(\alpha-\beta+\frac{1}{4},\alpha+\beta+\frac{1}{4};1+2\alpha;y\right)\label{eq:sol}
% \end{gather}
% with
% \begin{gather}
%   \alpha=\frac{1}{4}\sqrt{k^2-8 k+8}\\
%   \beta=\beta(\lambda)=\frac{1}{4}\sqrt{(k-1)^2-4\lambda}.
% \end{gather}
% As we restrict ourselves to the real solutions we choose $c_1$ and
% $c_2$ in such a way that $g$ is real. Without further conditions,
% $\lambda$ can be any real number.


% The expansion of the solutions at $y=0$ yields
% \begin{gather}
%   \phi_1(y)=y^{\frac{-4\alpha-k+2}{4}}(1+O(y))\\
%   \phi_2(y)=y^{\frac{4\alpha-k+2}{4}}(1+O(y))
% \end{gather}


Stability of such mapping can be studied by investigating the energy
of linear perturbations of solutions of \eqref{eq:el}. The energy of
such perturbation is given by
\begin{gather}\label{eq:infE}
  \epsilon(h)=\delta^2E(f)(h,h)=\frac{1}{2}V(S^{k-1})\int_0^{\pi}\bigg(h'^2+(k-1)\frac{\cos
    2f}{\sin^2\psi }h^2\bigg)\sin^{k-1}\psi d\psi.
\end{gather}





The domain of $\epsilon$ is the set of functions for which
$\epsilon<\infty$, we will further denote such set
$\mathcal{D}(\epsilon)$ will further be denoted as $D_k$.


If we consider the lowest power of the Laurent series expansion of $h$
to be $\alpha$, the square integrability of its derivative requires
$\alpha>\alpha_{\text{crit}}$, where $\alpha_{\text{crit}}=1-k/2$. The
domain $D_k$ can be split into the symmetric and antisymmetric parts
without the loss of generality, such parts will be denoted as $D^+_k$
and $D^-_k$ respectively, which in turn lets us restrict the interval
$[0,\pi]$ to its half: $[0,\pi/2]$. Direct calculation gives us
\begin{gather}
  \epsilon(h)=-\frac{1}{2}(h,Lh)+\frac{1}{2}hh'\sin^{k-1}\psi\bigg|_0
\end{gather}
with
\begin{gather}
  (f,g)=\frac{1}{2}V(S^{k-1})\int_0^\pi f(\psi) g(\psi) \sin^{k-1}\psi d\psi,\\
  Lh=\frac{1}{\sin^{k-1}\psi}\big(h'\sin^{k-1}\psi\big)'-(k-1)\frac{\cos 2 f}{\sin^2\psi}h,\\
  (Lg,f)-(g,Lf)=gf\left(\log\frac{g}{f}\right)'\sin^{k-1}\psi\bigg|_0.
\end{gather}
We have left the boundary terms (with the value at $\pi/2$ equal to
$0$ for symmetric/antisymmetric subspaces) because they shall play an
important role in further discussion.

and the question of stability is brought to the eigenvalue problem for
symmetric operator $L$.\\

The mapping of our interest is $f(\psi)=\frac{\pi}{2}$, the eigenvalue
problem states that
\begin{gather}\label{eq:problem}
  (L-\lambda)h=h''+(k-1)\cot(\psi)h'+\left(\frac{k-1}{\sin^2\psi}-\lambda\right)h=0.
\end{gather}
The substitution $y=\sin^2\psi$, $g(y)=h(\arcsin(y^{1/2}))$ reveals
the Fuchsian type equation with three regular singular points at $0$,
$1$ and $\infty$:
\begin{gather}
  4y(y-1)g''+2(y+k(y-1))g'+(\lambda-\frac{k-1}{y})g=0.
\end{gather}
The general solution of the latter is of the form
\begin{gather}
  g=c_1\phi_1+c_2\phi_2\\
  \phi_1(y)=y^{\frac{-4\alpha-k+2}{4}}{}_2F_1 \left(-\alpha-\beta+\frac{1}{4},-\alpha+\beta+\frac{1}{4};1-2\alpha;y\right)\\
  \phi_2(y)=y^{\frac{+4\alpha-k+2}{4}}{}_2F_1 \left(\alpha-\beta+\frac{1}{4},\alpha+\beta+\frac{1}{4};1+2\alpha;y\right)\label{eq:sol}
\end{gather}
with
\begin{gather}
  \alpha=\frac{1}{4}\sqrt{k^2-8 k+8}\\
  \beta=\beta(\lambda)=\frac{1}{4}\sqrt{(k-1)^2-4\lambda}.
\end{gather}
First and foremost we are interested in the behavior at
$\psi\rightarrow0$. The series expansion reveals that
\begin{gather}
  \phi_1(y)=y^{\frac{-4\alpha-k+2}{4}}(1+O(y))\\
  \phi_2(y)=y^{\frac{4\alpha-k+2}{4}}(1+O(y))
\end{gather}
% First, one should notice that, for $\Im(\alpha)\ne0$, $y^\alpha$ is
% not integrable on $[0,1]$. This introduces the condition
% $k>2(2+\sqrt{2})$ or $k<2(2-\sqrt{2})$. The latter case shall not be
% considered as we assume $k\ge3$.
For $k>2(2+\sqrt{2})=k_{\text{crit}}\approx6.8284$ we have
$\frac{-4\alpha-k+2}{4}<\frac{\alpha_{\text{crit}}}{2}$, so for
$k>k_{\text{crit}}$ the first solution has to be discarded, also for
such $k$, we have $\Im{\alpha}=0$, thus there are no oscillations
around $y=0$. This makes the case $k>k_{\text{crit}}$ much easier to
solve and we shall take care of it in the following section.

\section*{Stability for $k>k_{\text{crit}}$}

At $y\rightarrow1$ the values of $h$ and
$h'$ are
\begin{align}
  h=g&
  \sim\frac{\Gamma(1+2\alpha)}{\Gamma\left(\frac{3}{4}+\alpha-\beta\right)\Gamma\left(\frac{3}{4}+\alpha+\beta\right)}+O(1-y),\\
  h'=2y^{1/2}(1-y)^{1/2}g'&
  \sim\frac{\left(\left(\frac{1}{4}+\alpha\right)^2-\beta^2\right)\Gamma(2+2\alpha)}{\Gamma\left(\frac{5}{4}+\alpha-\beta\right)\Gamma\left(\frac{5}{4}+\alpha+\beta\right)}+O(1-y)
\end{align}
The antisymmetry of $h$, requires that $h(\pi/2)=g(1)=0$, and this is
true iff $\beta^2=(m_1+3/4+\alpha)^2$ for
$m_1\in\mathbb{N}_0$. Similarly if $h'(\pi/2)$ vanishes then
necessarily $\beta^2=(m_2+5/4+\alpha)^2$ for $m_2\in\mathbb{N}_0$ (the
special case when numerator is zero is covered by the $m_2=1$). Both
formulas can be combined into $\beta^2=(\alpha+(3+2n)/4)^2$ for
$n\in\mathbb{N}_0$ with $n$ even for symmetric modes and odd for
antisymmetric ones. We can calculate $\lambda_n$ as
\begin{align}
  \lambda_n&=\frac{1}{2}(-\alpha(8n+12)+(3k-2n^2-6n-8))\\
  &=\frac{1}{4}(2n+4-k+\alpha)(2n+2+k+\alpha)
\end{align}

The noticeable fact is, that the only eigenvalues grater than zero are
for $n=0$ (antisymmetric solution), and as $k\rightarrow\infty$ one
has $\lambda_0= 2 + O(1/k)$ and $\lambda_1=-k+O(1)$. More generally,
the asymptotic of $\lambda$ w.r.t. $k$ is as follows
\begin{gather}
  \lambda_n=-nk+(n+1)(2-n)+O(1/k).
\end{gather}
(TODO: mention the special dimension $k=7$ where $\lambda_1=0$.)
Substituting eigenvalues to \eqref{eq:sol} and using the value of
\begin{gather}
%   \beta^-_m=3/4+m+\alpha\\
%   \beta^+_m=1/4+m+\alpha\\
  \beta_n=\beta(\lambda_n)=\frac{1}{4}(3+2n+4\alpha)
\end{gather}
we end up with the complete solution to \eqref{eq:problem}
\begin{gather}
%   h^-_m(\psi)=A\sin^{2\alpha+\frac{-k+2}{2}}\psi\sideset{_2^{}}{_1^{}}F(-\frac{1}{2}-m,1+m+2\alpha;1+2\alpha;\sin^2\psi),\\
%   h^+_m(\psi)=B\sin^{2\alpha+\frac{-k+2}{2}}\psi\sideset{_2^{}}{_1^{}}F(-m,\frac{1}{2}+m+2\alpha;1+2\alpha;\sin^2\psi),\\
  h_n(\psi)=C_n\sin^{\frac{4\alpha-k+2}{2}}\psi {}_2 F_1
  \left(-\frac{n+1}{2},\frac{n+1}{2}+\frac{1+4\alpha}{2};1+2\alpha;\sin^2\psi\right).
\end{gather}


For $k=7$ the solution turns out to be particularly simple, because the
factor $(1+4\alpha)/2=1$, and we can use the compactified form of
hypergeometric function
\begin{gather}
%   h^-_m(\psi)=A_m\sin^{-2}\psi \sideset{_2^{}}{_1^{}}F(-\frac{1}{2}-m,\frac{3}{2}+m;\frac{3}{2};\sin^2\psi)
%   =\frac{\sin((2+2m)\psi)}{(2+2m)\sin^3\psi},\\
%   h^+_m(\psi)=B_m\sin^{-2}\psi \sideset{_2^{}}{_1^{}}F(-m,1+m;\frac{3}{2};\sin^2\psi)
%   =\frac{\sin((1+2m)\psi)}{(1+2m)\sin^3\psi}.
  h_n(\psi)=C_n\sin^{-2}\psi
  {}_2F_1\left(-\frac{n+1}{2},\frac{n+1}{2}+1;\frac{3}{2};\sin^2\psi\right)
  =\frac{C_n}{n+2}\frac{\sin((n+2)\psi)}{\sin^3\psi},\\
  \lambda_n=(5+n)(1-n).
\end{gather}

% which can be compactified to

% \begin{gather}
%   h_k(\psi)=C_k\frac{\sin(k\psi)}{k\sin^3\psi}
% \end{gather}


\section*{Stability for $k<k_{\text{crit}}$}

$\phi_1=\bar{\phi_2}$ for $k<k_\text{crit}$ and $\lambda\in R$ (from
$\overline{F(a,b;c;y)}=F(\bar{a},\bar{b};\bar{c};\bar{y})$, and the
symmetry in $a$ and $b$, using the fact, that $\beta$ is purely real
or purely imaginary), with both, $\phi_1$ and $\phi_2$ integrable. For
$k>k_\text{crit}$ $\phi_2=\bar{\phi_2}$, and $\phi_1$ is not
integrable, the most general form of the real solution when
$k<k_\text{crit}$ is (with $\phi:=\phi_1$, and $a_1$, $a_2$ - real
parameters)
\begin{gather}
  g=a_1(\phi+\bar\phi)+i a_2(\phi-\bar\phi)
\end{gather}
For $L$ to be self-adjoint the boundary terms are of the form
(in $f$,$g$ - functions of $\psi$)
\begin{gather}
  (f'g-fg')\sin^{k-1}\psi\big|^\pi_0\sim \bigg(\ln \frac{f}{g}\bigg)'fg\sin^{k-1}\psi\big|_0^\pi
\end{gather}
and have to vanish, which in the case of $k<k_\text{crit}$ reduces to
the requirement, that all eigenfunctions have linearly-dependent
asymptotics at $0+$ (this can be achieved, because the leading
asymptotics at $0+$ does not depend on $\lambda$) and, as $a1$ and
$a2$ are chosen globally, for the whole domain of $g$, the domain of
self-adjoint operator $L$ is therefore parametrized by the ``phase''
parameter $\theta\in[0,\pi]$ of $(a_1,a_2)$, which is an element of
$RP^1\sim^?U(1)$ (which agrees with the von Neuman theorem for
non-unique self-adjoint extensions), and the behaviour of the solutions
for fixed $\theta$ are dependent of $\lambda$ only. This means that
for given $\theta$ we can quantize $\lambda$ using boundary conditions
as for $k>k_\text{crit}$ (symmetricity/antisymmetricity). This leaves
us with the family of sets of eigenvalues $\{\lambda_m\}$ parametrized
by the $\theta$. Conversely, for arbitrary large $\lambda$, there
exists $\theta$ for which the boundary values are fulfilled (unless,
$\phi(1)\ne0$). This becomes obvious if one uses the relation for
$g_\lambda(1)$ and $\theta$:
\begin{gather}
  g_\lambda(1)=a_0(\phi_\lambda(1)
  e^{i\theta}+\overline{\phi_\lambda(1)} e^{-i\theta})=2a_0\Re
  (\phi_\lambda(1) e^{i\theta})=0
\end{gather}
and from continuity of $\phi_\lambda(1)$ in $\lambda$ two situations
may occur in the limit of $\lambda\rightarrow\infty$. Either, there is
a limiting value of $\theta_0$ (from which it results in self-adjoint
bounded from above extensions ), or there is no limiting value of
$\theta$ in which case $\theta$ is expected to go round the $[0,\pi]$
interval infinitely many times in which case arbitrary $\theta$ is
``crossed'' infinitely many times with the value of $\lambda$
increasing at each round, and we and up with extension unbound from
top for any $\theta$. The unboundness from below can be shown in the
same way of reasoning, or using the Sturm-Liouville theorem, but is of
no particular use.

%%% Local Variables:
%%% mode: latex
%%% TeX-master: "master"
%%% End:

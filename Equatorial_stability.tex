\section*{Stability of equatorial map}

For $f$ to extremalize the energy it is required that
\begin{gather}\label{eq:el}
  \frac{1}{\sin^{k-1}\psi}\big(\sin^{k-1}\psi f'\big)'-\frac{k-1}{2}\frac{\sin 2 f}{\sin^2\psi}=0.
\end{gather}
Stability of such mapping can be studied by investigating the energy
of linear perturbations of solutions of \eqref{eq:el}. The energy of
such perturbation is given by
\begin{gather}\label{eq:infE}
  \epsilon(h)=\delta^2E(f)(h,h)=\frac{1}{2}V(S^{k-1})\int_0^{\pi}\bigg(h'^2+(k-1)\frac{\cos 2f}{\sin^2\psi }h^2\bigg)\sin^{k-1}\psi d\psi.
\end{gather}
We set the domain of $\epsilon$ to be the set of twice differentiable
functions, square integrable w.r.t. the weight
$w(\psi)=\sin^{k-1}\psi$, which will further be denoted as $H_k$. If
we consider the lowest power of the Laurent series expansion of $h$ to
be $\alpha$, the square integrability of its derivative requires
$\alpha>\alpha_{\text{crit}}$, where $\alpha_{\text{crit}}=1-k/2$. On
the other hand, the symmetry $\psi\rightarrow\pi-\psi$ of
\eqref{eq:baseE} implies the symmetries $h\rightarrow h$ and
$h\rightarrow-h$, so the domain of $H_k$ can be split into the
symmetric and antisymmetric parts without the loss of generality, such
parts will be denoted as $H^+_k$ and $H^-_k$ respectively, which in
turn lets us restrict the interval $[0,\pi]$ to its half:
$[0,\pi/2]$. Moreover it is easy to show by a direct calculation, that
\begin{gather}
  \epsilon(h)=-(h,Lh)
\end{gather}
with
\begin{gather}
  (f,g)=\frac{1}{2}V(S^{k-1})\int_0^\pi f(\psi) g(\psi) \sin^{k-1}\psi d\psi\\
  Lh=\frac{1}{\sin^{k-1}\psi}\big(h'\sin^{k-1}\psi\big)'-(k-1)\frac{\cos 2 f}{\sin^2\psi}h\\
  (g,Lf)-(Lg,f)=0
\end{gather}
and the question of stability is brought to the eigenvalue problem for
self-adjoint operator $L$.\\

The mapping of our interest is $f(\psi)=\frac{\pi}{2}$, the eigenvalue
problem states that
\begin{gather}\label{eq:problem}
  (L-\lambda)h=h''+(k-1)\cot(\psi)h'+((k-1)\frac{1}{\sin^2\psi}-\lambda)h=0.
\end{gather}
The substitution $y=\sin^2\psi$, $g(y)=h(\arcsin(y^{1/2}))$ reveals
the Fuchsian type equation with three regular singular points at $0$,
$1$ and $\infty$:
\begin{gather}
  4y(y-1)g''+2(y+k(y-1))g'+(\lambda-\frac{k-1}{y})g=0.
\end{gather}
The general solution of the latter is of the form
\begin{gather}
  g=c_1\phi_1+c_2\phi_2\\
  \phi_1(y)=y^{\frac{-4\alpha-k+2}{4}}{}_2F_1 (-\alpha-\beta+\frac{1}{4},-\alpha+\beta+\frac{1}{4};1-2\alpha;y)\\
  \phi_2(y)=y^{\frac{+4\alpha-k+2}{4}}{}_2F_1 (\alpha-\beta+\frac{1}{4},\alpha+\beta+\frac{1}{4};1+2\alpha;y)\label{eq:sol}
\end{gather}
with
\begin{gather}
  \alpha=\frac{1}{4}\sqrt{k^2-8 k+8}\\
  \beta=\beta(\lambda)=\frac{1}{4}\sqrt{(k-1)^2-4\lambda}.
\end{gather}
First and foremost we are interested in the behavior at
$\psi\rightarrow0$. The series expansion reveals that
\begin{gather}
  \phi_1(y)=y^{\frac{-4\alpha-k+2}{4}}(1+O(y))\\
  \phi_2(y)=y^{\frac{4\alpha-k+2}{4}}(1+O(y))
\end{gather}
First, one should notice that, for $\Im(\alpha)\ne0$, $y^\alpha$ is
not integrable on $[0,1]$. This introduces the condition
$k>2(2+\sqrt{2})$ or $k<2(2-\sqrt{2})$. The latter case shall not be
considered as we assume $k\ge3$. Moreover, for $k>2(2+\sqrt{2})$ we
have $\frac{-\alpha-k+2}{4}<\frac{\alpha_{\text{crit}}}{2}$, so the
first solution will be discarded. At $y\rightarrow1$ the values of $h$
and $h'$ are
\begin{gather}
  h=g\sim\frac{\Gamma(1+2\alpha)}{\Gamma(\alpha-\beta)\Gamma(\alpha+\beta)}+O(1-y),\\
  h'=2y^{1/2}(1-y)^{1/2}g'\sim\frac{\Gamma(2+2\alpha)}{\Gamma(\alpha-\beta+1)\Gamma(\alpha+\beta+1)}+O(1-y).
\end{gather}
The antisymmetry of $h$, requires that $h(\pi/2)=g(1)=0$, and this is true
iff $\alpha-\beta=-m$ or $\alpha+\beta=-m$ for
$m\in\mathbb{N}_0$. Both conditions give rise to the same following
quantization of $\lambda^-$
\begin{gather}
  \lambda^-_m=2\alpha (-4 m-3)+\frac{1}{2}(3 k-8
  m^2-12 m-8).
\end{gather}
Analogously for symmetric solution
\begin{gather}
  \lambda^+_m=2\alpha(-4 m-1)+\frac{1}{2} (3 k-8
  m^2-4 m-4).
\end{gather}
With $\lambda_n$ defined as
\begin{gather}
  \lambda_n=\frac{1}{2}\big(-4\alpha(2n+1)+
  (3k-2n^2-2n-4)\big)
\end{gather}
for $n\in\mathbb{N}_0$, those two families of eigenvalues can be combined
using the relation
\begin{gather}
  \lambda_{2m}=\lambda_m^+\\
  \lambda_{2m+1}=\lambda_m^-
\end{gather}
The noticeable fact is, that the only eigenvalues grater than zero are
for $n=0$ (symmetric) and $n=1$ (antisymmetric), and as
$k\rightarrow\infty$ one has $\lambda_0= k + O(1/k)\rightarrow\infty$
and $\lambda_1\rightarrow 2$. More generally, the asymptotic of
$\lambda$ w.r.t. $k$ is as follows
\begin{gather}
  \lambda_n=(1-n)k+n(3-n)+O(1/k).
\end{gather}
Substituting eigenvalues to \eqref{eq:sol} and using the value of
\begin{gather}
%   \beta^-_m=3/4+m+\alpha\\
%   \beta^+_m=1/4+m+\alpha\\
  \beta_n=\beta(\lambda_n)=\frac{1}{4}(2n+1+4\alpha)
\end{gather}
we end up with the complete solution to \eqref{eq:problem}
\begin{gather}
%   h^-_m(\psi)=A\sin^{2\alpha+\frac{-k+2}{2}}\psi\sideset{_2^{}}{_1^{}}F(-\frac{1}{2}-m,1+m+2\alpha;1+2\alpha;\sin^2\psi),\\
%   h^+_m(\psi)=B\sin^{2\alpha+\frac{-k+2}{2}}\psi\sideset{_2^{}}{_1^{}}F(-m,\frac{1}{2}+m+2\alpha;1+2\alpha;\sin^2\psi),\\
  h_n(\psi)=C_n\sin^{\frac{4\alpha-k+2}{2}}\psi {}_2 F_1 (-\frac{n}{2},\frac{n}{2}+\frac{1+4\alpha}{2};1+2\alpha;\sin^2\psi).
\end{gather}


For $k=7$ the solution turns out to be particularly simple, because the
factor $(1+4\alpha)/2=1$, and we can use the compactified form of
hypergeometric function
\begin{gather}
%   h^-_m(\psi)=A_m\sin^{-2}\psi \sideset{_2^{}}{_1^{}}F(-\frac{1}{2}-m,\frac{3}{2}+m;\frac{3}{2};\sin^2\psi)
%   =\frac{\sin((2+2m)\psi)}{(2+2m)\sin^3\psi},\\
%   h^+_m(\psi)=B_m\sin^{-2}\psi \sideset{_2^{}}{_1^{}}F(-m,1+m;\frac{3}{2};\sin^2\psi)
%   =\frac{\sin((1+2m)\psi)}{(1+2m)\sin^3\psi}.
  h_n(\psi)=C_n\sin^{-2}\psi
  {}_2F_1(-\frac{n}{2},\frac{n}{2}+1;\frac{3}{2};\sin^2\psi)
  =C_n\frac{\sin((1+n)\psi)}{(1+n)\sin^3\psi}.
\end{gather}

% which can be compactified to

% \begin{gather}
%   h_k(\psi)=C_k\frac{\sin(k\psi)}{k\sin^3\psi}
% \end{gather}


\section{Stability below $k_{crit}$}
\label{sec:Notes}

$\phi_1=\bar{\phi_2}$ for $k<k_\text{crit}$ and $\lambda\in R$ (from
$\overline{F(a,b;c;y)}=F(\bar{a},\bar{b};\bar{c};\bar{y})$, and the
symmetry in $a$ and $b$, using the fact, that $\beta$ is purely real
or purely imaginary), with both, $\phi_1$ and $\phi_2$ integrable. For
$k>k_\text{crit}$ $\phi_2=\bar{\phi_2}$, and $\phi_1$ is not
integrable, the most general form of the real solution when
$k<k_\text{crit}$ is (with $\phi:=\phi_1$, and $a_1$, $a_2$ - real
parameters)
\begin{gather}
  g=a_1(\phi+\bar\phi)+i a_2(\phi-\bar\phi)
\end{gather}
For $L$ to be self-adjoint the boundary terms are of the form
(in $f$,$g$ - functions of $\psi$)
\begin{gather}
  (f'g-fg')\sin^{k-1}\psi\big|^\pi_0\sim \bigg(\ln \frac{f}{g}\bigg)'fg\sin^{k-1}\psi\big|_0^\pi
\end{gather}
and have to vanish, which in the case of $k<k_\text{crit}$ reduces to
the requirement, that all eigenfunctions have lineary-dependent
asymptotics at $0+$ (this can be achieved, because the leading
asymptotics at $0+$ does not depend on $\lambda$) and, as $a1$ and
$a2$ are chosen globally, for the whole domain of $g$, the domain of
self-adjoint operator $L$ is therefore parametrized by the ``phase''
parameter $\theta\in[0,\pi]$ of $(a_1,a_2)$, which is an element of
$RP^1\sim^?U(1)$ (which agrees with the von Neuman theorem for
non-unique self-adjoint extensions), and the behaviour of the solutions
for fixed $\theta$ are dependent of $\lambda$ only. This means that
for given $\theta$ we can quantize $\lambda$ using boundary conditions
as for $k>k_\text{crit}$ (symmetricity/antisymmetricity). This leaves
us with the family of sets of eigenvalues $\{\lambda_m\}$ parametrized
by the $\theta$. Conversely, for arbitrary large $\lambda$, there
exists $\theta$ for which the boundary values are fulfilled (unless,
$\phi(1)\ne0$). This becomes obvious if one uses the relation for
$g_\lambda(1)$ and $\theta$:
\begin{gather}
  g_\lambda(1)=a_0(\phi_\lambda(1)
  e^{i\theta}+\overline{\phi_\lambda(1)} e^{-i\theta})=2a_0\Re
  (\phi_\lambda(1) e^{i\theta})=0
\end{gather}
and from continuity of $\phi_\lambda(1)$ in $\lambda$ two situations
may occur in the limit of $\lambda\rightarrow\infty$. Either, there is
a limiting value of $\theta_0$ (from which it results in self-adjoint
bounded from above extensions ), or there is no limiting value of
$\theta$ in which case $\theta$ is expected to go round the $[0,\pi]$
interval infinitely many times in which case arbitrary $\theta$ is
``crossed'' infinitely many times with the value of $\lambda$
increasing at each round, and we and up with extension unbound from
top for any $\theta$. The unboundness from below can be shown in the
same way of reasoning, or using the Sturm-Liouville theorem, but is of
no particular use.
\\
Knowing explicitly how $\phi_\lambda$ depends of $\lambda$, one can
calculate asymptotic behaviour of $\arg(\phi_\lambda(1))$ as
$\lambda\rightarrow\infty$. (tu policzylem przy pomocy Mathematici, ze
$\arg(\phi(1))\rightarrow\log(\lambda)$, wiec sie nie zbiega do zadnej
wartosci, ale wstrzymam sie ze szczegolami, zanim nie policze tego
wlasnorecznie).

\section{Self-similar ansatz}


Equation for gradient flow for the map considered is
\begin{equation}
  \label{eq:grad_flow}
  \partial_t f=f''+(k-1)\cot \psi f'-\frac{k-1}{2}\frac{\sin2f}{\sin^2 \psi})\\
  f\in [0,\pi]\\
  \psi\in [0,\pi]
\end{equation}


For $\psi<<1$ we have
\begin{equation}
  \partial_t f=f''+(k-1)\frac{1}{\psi} f'-\frac{k-1}{2}\frac{\sin2f}{\psi^2})\\
  \psi\in [0,\infty]
\end{equation}

Introducing self similar variables $\xi=\frac{\psi}{\sqrt{T-t}}$ and
$\tau=-\log(T-t)$, $F(\xi,\tau)$ the above equation becomes
\begin{equation}
  \partial_\tau F=F''+((k-1)\frac{1}{\xi}-\frac{\xi}{2}) F'-\frac{k-1}{2}\frac{\sin2F}{\xi^2})
\end{equation}
with blowup solutions
\begin{equation}
  0=F''+((k-1)\frac{1}{\xi}-\frac{\xi}{2}) F'-\frac{k-1}{2}\frac{\sin2F}{\xi^2}).
\end{equation}
The existance of the monotonic solutions to the above equation with
boundary conditions $F(0)=0$ and $F(\infty)=\pi$ has been prooven by
A. Gastel. The linearized form of the above equation around the blowup
solution is
\begin{equation}
  \partial_\tau \epsilon=\epsilon''+((k-1)\frac{1}{\xi}-\frac{\xi}{2}) \epsilon'-(k-1)\frac{\cos2F}{\xi^2}\epsilon=L\epsilon
\end{equation}

By construction, one of the unstable modes is $\xi F'$ (J. Eggers and
M.A. Fontelos, p.R20).

% \begin{equation}
%   0=h''-(k-2)\tanh x h'+\frac{k-1}{2}\sin2h
% \end{equation}


% For $y=1/\cosh^2x$ becomes
% \begin{equation}
%   \label{eq:grad_flow_y}
%   \partial_t h=4(1-y)y h''+2(k(1-y)-y)h'+\frac{k-1}{2y}\sin2h,
% \end{equation}
% which for $x^2>>1$ (or equivalently $y<<1$) becomes
% \begin{equation}
%   \label{eq:grad_flow_y_small}
%   \partial_t h=4y h''+2k h'+\frac{k-1}{2y}\sin2h,
% \end{equation}
% for which the self-similar variables can be introduced by defining
% \begin{equation}
%   \label{eq:self_sim_vars}
%   \begin{split}
%     \tau&=-\ln(T-t)\\
%     \xi &=\frac{y}{T-t},
%   \end{split}
% \end{equation}
% then we have
% \begin{equation}
%   \label{eq:self_sim_flow}
%   \partial_\tau h= 4\xi h''+(2k-\xi) h'+\frac{k-1}{2\xi}\sin2h
% \end{equation}


%%% Local Variables:
%%% mode: latex
%%% TeX-master: "master"
%%% End:

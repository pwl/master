Dirichlet energy $E(u)=\frac{1}{2}\int_M \lVert \nabla u\rVert^2 dV_M$
and its stationary points are central parts of many physical
theories. We can generalize the Dirichlet energy to any map between
differentiable manifolds endowed with inner products. Stationary
points of such functional are called harmonic maps in analogy to
harmonic functions, the main matter of this thesis are harmonic maps
between spheres.
\\

There is a powerful method available in the search for critical point
of a functional $E$ called heat flow. The basic idea behind heat flow
is to define a continuous vector field on the domain of a functional,
such that the functional value decreases along its integral
curves. Therefore, starting with any point as an initial state we can
deform it along the vector field to reach a local minimum and thus, a
stationary point. The heat flow realizes this idea by using an inverse
gradient of $E$ as a vector field in which case integral curves are
formally defined as solutions to
\begin{align*}
  \partial_t F=-\delta E(F),\quad F(0)=F_0,
\end{align*}
where $F_0$ is a starting point and $t$ is a parameter along the
integral curve. The name ``heat flow'' is due to the fact that for the
Dirichlet energy of the form $E(u)=\frac{1}{2}\int_M \lVert \nabla
u\rVert^2 dV_M$ the heat flow turns out to be a heat
equation. Proceeding with this analogy we will henceforth refer to the
parameter $t$ as to time. One expects that the flow will
asymptotically converge to the stationary point of $E$ where the
gradient is zero. This approach has been successfully used by Eells
and Sampson \cite{Eells1964} to prove the existence of harmonic maps
to manifolds of non positive sectional curvature.
\\

If there are two local minima of $E$, by the continuity of the vector
field there exist points which do not flow to either of them, which
stands as a heuristic argument in favor of the existence of a
non-trivial saddle point of a functional. There is however a major
flaw in this reasoning -- in some cases the quasilinear parabolic
equations yield singularities in finite time. Actually in the
particular case of the maps between spheres one can prove using a
theorem by Struwe (Theorem \ref{thm:Struwe}) that the solution
to the heat flow will blow up for some finite $t$.\\

In this thesis we analyse possible asymptotic states and the mechanism
of the blow-up produced by the heat flow for maps between
$k$-dimensional spheres. As the existence of harmonic maps between
spheres has been proved by Bizoń and Chmaj \cite{Bizon1997}, and
Corlette and Wald \cite{Corlette2001} we focus on dynamical aspects of
the heat flow such as stability analysis and the blow-up.

We start by defining the Dirichlet Energy and stating some basic facts
about the harmonic maps in general. Next, we present a symmetric
ansatz to reduce the problem of maps between spheres to maps between
$S^1$ and we proceed by analysing the resulting quasilinear parabolic
partial differential equation in one dimension. The stability of
harmonic maps is analysed in section \ref{sec:stab-ground-stat} and
the blow-up mechanism is described in section \ref{sec:blow-up}.

Numerical results on solving the heat flow in the given ansatz are
presented in chapter \ref{cha:numerical-results} to confirm the
analytical results.



%%% Local Variables:
%%% mode: latex
%%% TeX-master: "master"
%%% End:

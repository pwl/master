The harmonic maps present very elegant generalization of the wide
class of functionals one can encounter in mathematical physics. The
stationary points of such functionals along with their stability
properties are of the critical importance and their existence often
presents an essential difficulty. The following chapter contains very
brief introduction to harmonic maps as well as their basic properties
and the formulation of the main problem -- harmonic maps between
spheres.
\\
There is however a powerful method available in the search for
critical points of a given functional called the gradient flow, which
allows one to determine the existence of non trivial stationary points
along with some qualitative information about the stability as a
byproduct. The general formulation of the gradient flow along with its
application to the maps between spheres are presented in chapter
\ref{cha:gradient-flow}.
\\
However, gradient flow has a major flaw -- as many non-linear
parabolic equations it can produce singularities which cease its
existence after a finite time. This happens if maps between spheres
are treated with the gradient flow. One finds that the map becomes
discontinuous at a point of the domain sphere in a finite time. Still
this undesirable behaviour can be stripped (extracted?) and analysed
which is done in chapter \ref{cha:numerical-results}.
\\
The major aim of this thesis is the prove of existence of a
saddle-type harmonic map between two spheres of dimension $k$ each,
using the gradient flow. As the existence of such maps has been proved
by Bizoń and Chmaj \cite{Bizon1997} as well as Corlette and Wald
\cite{Corlette2001} we shall focus on the dynamics generated by the
gradient flow putting the pressure on the stability analysis on the
first place. The numerical results are presented in chapter
\ref{cha:numerical-results}, where the implementation of the gradient
flow has been used to produce the same results as the non-dynamical
approach to the harmonic map.



%%% Local Variables:
%%% mode: latex
%%% TeX-master: "master"
%%% End:

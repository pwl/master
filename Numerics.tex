\section{Numerical realization of the bisection}
\label{sec:numer-real-mount}

The procedure described in \ref{sec:bisection} can be easily realized
by solving the PDE \eqref{eq:en_flow} numerically. For $H_+$, we start
by choosing $g_+$ to be
\begin{align}
  \label{eq:98}
  g_+(\psi)=\pi\sin\psi.
\end{align}
Then, the proper $A^*$ is found by bisection between the blow-up and
the convergence to $0$. As the bisection cannot yield an exact value
of $A^*$ (due to the finite machine precision) we are not able to
reach precisely the saddle point. Rather, as we are starting a bit off
the mid-line between attractors, the solution slides along it reaching
the neighbourhood of $f_2$ along its stable direction, it stays put
for some time but then starts to move again, slowly decaying along the
unstable direction to finally either fall into $0$ or blow up. The
smaller the numerical error of $A^*$ the closer we get to the mid-line
and the longer solutions stays near $f_2$.\\

From the numerical solutions to PDE \eqref{eq:en_flow} we can read the
quantities involved in approaching and leaving the neighbourhood of
$f_2$. These are the first two modes of $f_2$ along with their
respective eigenvalues as presented in \eqref{eq:69}. To obtain the
eigenvalues $\lb{2}{0}$ and $\lb{2}{2}$ we use the function
$\partial_tf\big|_{r=\pi/2}$ which, while in a close neighbourhood of
$f_2$ is
\begin{align}
  \label{eq:99}
  \partial_t
  f\big|_{r=\pi/2}=-\lambda_0A_0e^{-\lambda_0t}v_0(\pi/2)-\lambda_2A_2e^{-\lambda_2t}v_2(\pi/2)
\end{align}
(we have dropped the upper index for clarity). $v_0(\pi/2)$ and
$v_2(\pi/2)$ are quantities depending on the normalization of our
choice, and to further simplify the calculations we set
\begin{align}
  \label{eq:100}
  v_{0,2}(\pi/2)=1.
\end{align}
The quantities $A_{0,2}$ and $\lb{2}{0,2}$ resulting from fitting the
function \eqref{eq:99} to its analog from numerically solved PDE along
with $A^*$ calculated from the bisection are presented in Table ???.\\

Figures ??? present the stages of the evolution.

% \begin{sideways}
%     \centering
%     \label{fig:Evol2}

% \hspace*{-1.5in}
\begin{figure}[h]
  \label{fig:snapshot2}
  \centering
  \advance\leftskip-3cm
  \input{./graphics/snapshot_to_multiplot.tex}
  \caption{This figure contains a sequence of snapshots of numerical
    solutions to the gradient flow equation with initial data
    $g_-(\psi)=\psi+B^*\cdot \sin(2\psi)$ with $B^*=...$. Each
    snapshot depicts $\lvert \partial_t f\rvert$ normalized so that
    $\lvert \partial_{tr} f(t,\pi/2)\rvert=1$ along with the plot of
    $f(t)$ in the upper right corner. For comparison the first two
    modes of $f_3$ (blue dashed lines), normalized to
    $v_{1,3}^{(3)\prime}(\pi/2)=1$ have been also depicted. The
    evolution can now be divided into the following stages: ($t=0$)
    non-linear evolution, ($t=1.5-2$) linear convergence to $f_3$
    along $\vn{3}{3}$, ($t=2.5-3.$) linear divergence along
    $\vn{1}{3}$, ($t=3.5$) non-linear approach to ground state $f_1$,
    ($t\ge4$) linear convergence to $f_1$ along $\vn{1}{1}$.}
\end{figure}
    % \caption{The first six non-trivial solutions to}
    %   \eqref{eq:f_psi_EL}}
% \end{sideways}

%%% Local Variables:
%%% mode: latex
%%% TeX-master: "master"
%%% End:

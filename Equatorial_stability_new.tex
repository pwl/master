\section{Stability of the equatorial map}
\label{sec:stab-equat-map}

The stability of the equatorial map is governed by the eigenvalues of
\begin{align}
  \label{eq:4}
  -v''-(k-1)\cot\psi v'-\frac{k-1}{\sin^2\psi}v=\Lambda v.
\end{align}
According to our convention, $\Lambda<0$ corresponds to unstable
direction. The above equation can be considered as an eigenproblem of
the operator $L$ such that
\begin{align}
  \label{eq:5}
  Lv=-v''-(k-1)\cot\psi v'-\frac{k-1}{\sin^2\psi}v.\\
  \D{L}=\{C^\infty_0(0,\pi)\}
\end{align}
(TODO:explain $L$ symmetric) where $C^\infty_0(0,\pi)$ is the set of
infinitely differentiable functions with support away from $0$ and
$\pi$. The Hilbert space in this case is
$L^2([0,\pi],\sin^{k-1}\psi)$. We can utilize the unitary map
$U:L^2([0,\pi],\sin^{k-1}\psi)\ra L^2([0,\pi])$ such that
$U:f(\psi)\ra f(\psi)\sin^{(k-1)/2}\psi$. $U$ maps $C_0^\infty(0,\pi)$
onto itself and transforms the middle term in \eqref{eq:5} into the
potential term
\begin{align}
  \label{eq:7}
  ULU^{-1}=\left(
    -\frac{d^2}{d\psi^2}+V(\psi)-\frac{3}{2}(k-1)\right)\\
  V(\psi)=\frac{1}{4}(k-7)(k-1)\cot^2\psi.
\end{align}
(Unitary maps leave the eigenvalues $\Lambda$ untouched.) We will get
rid of the constant potential term by defining
\begin{align}
  \label{eq:8}
  \lambda=\Lambda+\frac{3}{2}(k-1).
\end{align}
The eigenproblem changed its form to
\begin{align}
  \label{eq:9}
  Kf=\lambda f
\end{align}
or
\begin{align}
  \label{eq:15}
  -f''+V(\psi)f=\lambda f.
\end{align}
Now as the proper dynamics may be generated only by the self-adjoint
operators we will have to verify weather $K$ (and thus $L$) is a
self-adjoint operator.
\begin{theorem}
  $K$ is essentially self-adjoint iff $4-2\sqrt{3}\ge
  k\ge4+2\sqrt{3}$.
\end{theorem}
\begin{proof}
  We will use the Weyl's limit-point/limit-circle criterion to
  determine the deficiency indices of $K$ (see [Simon,Reed,theorem
  X.7]). To do that, we have to investigate the number of solutions to
  the equations
  \begin{align}
    \label{eq:12}
    -f''+Vf=if\\
    -f''+Vf=-if
  \end{align}
  As $K$ commutes with complex conjugation, by the von Newmann's
  theorem, it has equal deficiency indices and it is enough to
  consider solutions to
  \begin{align}
    \label{eq:14}
    -f''+Vf=if
  \end{align}
  with $f\in\LSq$. As $V$ is infinitely differentiable at every point
  $\psi$ from $(0,\pi)$, it is enough to check the number of
  admittable solutions at $\psi=0$ and at $\psi=\pi$. But the
  potential $V$ is symmetric under the reflection $\psi\ra\pi-\psi$
  and thus so is $K$, therefore the exponents at $\psi=0$ and at
  $\psi=\pi$ are the same. At $\psi=0$ the possible solutions to
  \eqref{eq:14} behave as $\psi^{\gamma_+}$ and $\psi^{\gamma_-}$ with
  \begin{align}
    \label{eq:10}
    \gamma_\pm=\frac{1}{2}\left(1\pm\sqrt{k^2-8k+8}\right).
  \end{align}
  For $\psi^{\gamma_-}$ not to be in $\LSq$ we need $4-2\sqrt{3}\ge
  k\ge4+2\sqrt{3}$, so for such $k$ there would be only one solution
  to \eqref{eq:14} at each boundary. By the Weyl's limit-point,
  limit-circle criterion we obtain the claimed theorem.
\end{proof}
As the byproduct of this proof we get the following corollary
\begin{corollary}
  The deficiency indices of $K$ are $<2,2>$ for
  $k\in(4-2\sqrt{3},4+2\sqrt{3})$.
\end{corollary}
The exponents of the solutions to \eqref{eq:15} at $\psi=0$ can be
written in the more compact form by defining
\begin{align}
  \label{eq:18}
  \alpha=\frac{1}{2}\sqrt{k^2-8k+8},\\
  \gamma_\pm=\frac{1}{2}\pm\alpha.
\end{align}
As $\alpha$ can be either real or purely imaginary depending on $k$,
and the potential $V(\psi)$ is either unbounded from above, zero or
unbounded from below we shall further split our problem into the
following sets of $k$
\begin{align}
  \label{eq:20}
  \Gamma_1&=(-\infty,4-2\sqrt{3}]\cup[4+2\sqrt{3},\infty),\\
  \Gamma_2&=(4-2\sqrt{3},1]\cup[7,4+2\sqrt{3}),\\
  \Gamma_3&=(1,4-2\sqrt{2}]\cup[4+2\sqrt{2},7),\\
  \Gamma_4&=(4-2\sqrt{2},4+\sqrt{2}).
\end{align}
For $\Gamma_1$ $K$ is essentially self-adjoint, for $k\in\Gamma_2$ the
exponents are real and the potential is non-negative, within
$\Gamma_3$ the exponents are still real but the potential becomes
unbounded from below and for $k\in\Gamma_4$ the exponents have
imaginary parts which generate oscillations around the origins with
the potential still being unbounded from below.\\

As $\Gamma_i$ for different $i$ need different treatment we continue
the stability analysis for each of them in three separate sections.

%%% Local Variables:
%%% mode: latex
%%% TeX-master: "master"
%%% End:

\section{Gradient flow}
\label{sec:gradient-flow}

One can approach determining the properties of the critical points of
a functional by numerous ways. For example one can prove the existence
of solutions to Euler-Lagrange equations using some ODE techniques,
but this rarely gives any insight into the geometry underlying the
functional. Another more sophisticated way is to analyse the level
sets of the functional and use the infinite dimensional variant of the
Morse theory to obtain the number of critical points. There is however
a distinguishable technique called the gradient flow, which consists
of defining the flow, which solutions move along the gradient lines of
the functional in the direction of the negative gradient.\\


The intuition behind the gradient flow is that if we start with some
initial state, the gradient flow will push the solution into the
minimum of the functional (if the functional is bounded from below),
and if the flow exists for all times it will always obtain its
asymptotic state. This approach is especially useful for finding the
harmonic map of a given degree, because the degree is conserved during
the flow, so the flow can possibly deform any map into its homotopy
equivalent harmonic map. However, this scenario can hardly ever be
realized because the flow usually runs into singular solutions in
finite or infinite time. This is the case for Dirichlet
energy of the maps $S^k\ra S^k$.\\

The formal setup for the gradient flow of the Dirichlet energy
\eqref{eq:30} is
\begin{align}
  \label{eq:41}
    \begin{split}
    F\big|_{t=0}&=F_0\\
    \partial_t F^A&=-\frac{\delta E}{\delta F_A}=-L(F)^A
  \end{split}
\end{align}
Where $L$ is calculated to be
\begin{align}
  \label{eq:43}
  -L(F)^C=\Delta_g F^C+(\Gamma_{h(F)})_{AB}^{C}\frac{\partial
    F^A}{\partial x^a}\frac{\partial F^B}{\partial x^b}g^{ab}
\end{align}
and generally is a quasi-linear operator. By our assumption on the
domain and the target, \eqref{eq:41} is the set of parabolic
PDE's. The differentiation of \eqref{eq:41} w.r.t. $t$ gives
\begin{align}
  \label{eq:40}
  E_t(F)=-\int_M \partial_t F_A\partial_t F_B h^{AB} dV_M\le0
\end{align}
with $E_t(F)=0$ if and only if $F$ is harmonic. The flow \eqref{eq:41}
therefore reduces the energy and asymptotically tends to some
critical point of $E$.\\

The question if there exists the harmonic map of the same homotopy
class as the initial data $F_0$ is thus reduced to the existence of
the solution to \eqref{eq:41} for all $t$ which is not singular at
$t\ra\infty$. This method was used in \cite{Eells1964} to prove the
theorem \ref{thm:Eells-Sampson}, which now can be reformulated in the
following form.

\begin{theorem}[Eells-Sampson \cite{Eells1964}]\label{thm:Eells-Sampson2}
  If $N$ is compact and has non-positive Riemannian curvature, then
  for every $F_0$ the solution to \eqref{eq:41} exists for
  $t\in[0,\infty)$ and converges to a harmonic map for $t\ra\infty$.
\end{theorem}

As it has been mentioned, the solutions to \eqref{eq:41} are not
guaranteed to exists for arbitrary large times. The criterion which
guarantees that the solution will blow up at some time $T$ is given by
Struwe

\begin{theorem}[Struwe]\label{thm:Struwe}
  For any time $T>0$ there exists a constant
  $\epsilon=\epsilon(M,N,T)>0$ such that for any map $F_0:M\ra N$
  which is not homotopic to a constant and satisfies $E(F_0)<\epsilon$
  the solution $F$ to \eqref{eq:41} must blow up before time $2T$.
\end{theorem}

In the case when $M$ has non-positive Riemannian curvature the theorem
also applies but, within a given homotopy class, the smooth harmonic
map obtains the global minimum which is greater than the constant
$\epsilon(M,N,T)$, thus no initial data can fulfill the criterion
$E(F_0)<\epsilon$.

\subsection{Gradient flow for $S^k\ra S^k$}
\label{sec:gradient-flow-skra}

Before we write any evolution equations we can state two basic facts
about the Dirichlet energy \eqref{eq:36}. Due to the existence of the
one parameter family of conformal maps of a sphere we can decrease the
energy of a given map $F$ to arbitrary small values. The conformal map
in local coordinates used in section \ref{sec:basic-setup} has the
form
\begin{align}
  \label{eq:10}
  \psi_A=2\arctan(e^A\tan(\psi/2)).
\end{align}
The above conformal map can be depicted as dragging the whole sphere
along the longitudal coordinates in the direction of one of its poles
(for $A$ large, this would be the north pole). We define the map $F_A$
to be a composition
\begin{align}
  \label{eq:11}
  F_A=F\circ\psi_A.
\end{align}
Due to the conformal properties of the map $\psi_A$ we obtain the
following energy density of $F_A$
\begin{align}
  \label{eq:12}
  e(F_A)_\psi=&e(F)_{\psi_A}\rho_A^2,\\
  \rho_A=&\frac{1}{\cosh A-\cos\psi\sinh A},
\end{align}
where $e(F)_\psi$ is the energy density of $F$ at point $\psi$.  The
map $F$ is regular, and therefore $e(F)_{\psi_A}$ is bounded by its
maximal value $C(F)=\max_\psi\left(e(F)_\psi\right)$ and we have the
following bound
\begin{align}
  \label{eq:13}
  e(F_A)_\psi&\le C(F)\rho_A^2.
\end{align}
Assuming $k\ge3$, the Dirichlet energy of $F_A$ can be bounded by
\begin{align}
  \label{eq:14}
  E(F_A)&=\int_{S^k}e(F_A)dV_{S^k}\\
  &\le C(F)V(S^{k-1})\int_{0}^{\pi}\rho_A^2\sin^{k-1}\psi d\psi\\
  &\le C(F)V(S^{k-1})\max_\psi(\sin^{k-3}\psi)\int_{0}^{\pi}\rho_A^2\sin^2\psi d\psi\\
  &\le C_1(F,k)\frac{1}{1+\cosh A}.
\end{align}
which can be made arbitrary small by an appropriate choice of $A$. The
above bound is not optimal, but is enough for our purposes. As the map
$\psi_A$ leaves the north and south poles of $S^k$ unchanged, the map
$F_A$ is of the homotopy class that $F$ and we conclude by the
following theorem.

\begin{theorem}\label{thm:skk-energy-bound}
  For $k\ge3$ and any given map $F:S^k\ra S^k$ there is a map within
  the homotopy class of $F$ of arbitrary small Dirichlet energy.
\end{theorem}

The consequence of theorems \ref{thm:Struwe} and
\ref{thm:skk-energy-bound} is that in any homotopy class of maps
$S^k\ra S^k$ there are maps for which the gradient flow will yield a
singular behaviour. From this point we assume the map $F:S^k\ra S^k$
is
consistent with the ansatz \eqref{eq:24}.\\

From this point we will assume the map $F:S^k\ra S^k$ is consistent
with the ansatz \eqref{eq:24} and we shall consider the functional
$E(f)$ as defined in \eqref{eq:36}. We state the following.

\begin{theorem}
  For $k\ge3$ the only local minima of $E(f)$ are $f=n\pi$.
\end{theorem}
\begin{proof}
  To show the thesis it is enough to find a vector $v$ such that along
  $v$ the energy decreases, or equivalently
  \begin{align}
    \label{eq:18}
    \delta^2E(v,v)<0.
  \end{align}
  It turns out that the conformal Killing field
  \begin{align}
    \label{eq:20}
    K=\sin\psi\frac{\partial}{\partial \psi}.
  \end{align}
  which is the generator of \eqref{eq:10} has the required
  property. We now set $v=f'(\psi)\sin\psi$, for which
  \begin{align}
    \delta^2E(v,v)
    &=\int_0^{\pi}
    \left(
      v'^2+(k-1)\frac{\cos2f}{\sin^2\psi}v^2
    \right)\sin^{k-1}\psi d\psi\\
    &=-\int_0^{\pi}
    \left(\frac{1}{\sin^{k-1}\psi}\left(\sin^{k-1}\psi
        v'\right)'-(k-1)\frac{\cos2f}{\sin^2\psi}v\right)v\sin^{k-1}\psi
    d\psi\\
    &=(v,\mathcal L v).
  \end{align}
  From \eqref{eq:strange_variation} and by the fact that $f$ is the
  critical point we have
  \begin{align}
    \label{eq:21}
   \mathcal{L}v=(2-k)v
  \end{align}
  and thus
  \begin{align}
    \label{eq:22}
    \delta^2 E(v,v)=(2-k)\lVert v\rVert^2\le0.
  \end{align}
  This argument works for any non-constant $f$ which is the solution
  to \eqref{eq:f_psi_EL}. But the only constant regular stationary
  points $f=n\pi$ are the global minima of the Dirichlet energy which
  completes the proof.
\end{proof}

Applying the gradient flow to \eqref{eq:36} we obtain the following
Cauchy problem
\begin{equation}
  \label{eq:en_flow}
  \begin{split}
    f_t&=\frac{1}{\sin^{k-1}\psi}\left(\sin^{k-1}\psi
      f'\right)'-\frac{(k-1)}{2}\frac{\sin2f}{\sin^2\psi},\\
    f(0,\psi)&=f_0(\psi),\\
    f(t,0)&=0\quad f(t,\pi)=m\pi.
  \end{split}
\end{equation}


The choice of $m$ reduces the class of possible stationary points
to maps of homotopy degree up of $m$. Under evolution according to
\eqref{eq:40} the energy decreases
\begin{align}
  \label{eq:37}
  E_t(f)=-\int_0^{\pi}f_t^2\sin^{k-1}\psi d\psi
\end{align}
with $E_t(f)=0$ iff $f_t=0$. On the other hand, $E(f)\ge0$ so for
generic initial data we would expect energy to asymptotically decrease
to one of its local minima. Such minima can be located only at
stationary points which are not unstable under linear
perturbation. Due to the existence of conformal Killing field
$K=\sin(\psi)\frac{\partial}{\partial \psi}$ (with $\mathcal{L}_K
g=-2\cos\psi g$ TODO:check) the only local minima are constant
solutions $f_0=k\pi$.

% The parabolic equations of the form $f_t=L(f)$ (with $L$ being
% elliptic and possibly non-linear) have proved to be the useful tool
% for finding the solutions to the equation $L(f)=0$. In particular, if
% $L$ originates from the variation of some non-negative functional,
% let's say $E(f)\ge0$, its value decreases along the solutions to
% $f_t=L(f)$, i.e. $(E(f))_t\le0$, so the solutions of the evolution
% equation should formally converge to some point where $f_t=0$. We say
% formally, because for some choices of $L$, the evolution equation can
% lead to singular solutions in finite time, preventing the function $u$
% from approaching to the stationary point. We shall see that this is
% the case for \eqref{eq:36}.\\

%%% Local Variables:
%%% mode: latex
%%% TeX-master: "master"
%%% End:

\section{Gradient flow}
\label{sec:gradient-flow}


We will approach this problem with the means of the gradient
flow by considering the flow
\begin{equation}
  \label{eq:en_flow}
  \begin{split}
    f_t&=-\frac{\delta E}{\delta f}=\frac{1}{\sin^{k-1}\psi}\left(\sin^{k-1}\psi
      f'\right)'-\frac{(k-1)}{2}\frac{\sin2f}{\sin^2\psi}\\
    f\left(t,0\right)&=0\\
    f\left(t,\pi\right)&=m\pi\quad m\in\mathbb{N}
  \end{split}
\end{equation}
Such flow is also known as harmonic map flow because stationary points
of it coincide with the harmonic maps as they are solutions of the ODE
\eqref{eq:f_psi_EL}. The choice of $\delta$ reduces the class of
possible stationary points to maps of homotopy degree up to one. Under
evolution according to \eqref{eq:en_flow} the energy decreases
\begin{align}
  \label{eq:en_t}
  E_t(f)&=\lim_{h\ra0}\frac{E(f(t+h))-E(f(t))}{h}\\
  &=\lim_{h\ra0}\frac{E(f+h f_t+\mathcal{O}(h^2))-E(f)}{h}\\
  % =\lim_{h\ra0}\left(\frac{\delta E}{\delta f}(f_t)+\mathcal{O}h\right)\\
  &=\int_{-\pi/2}^{\pi/2}\frac{\delta E}{\delta f}f_t\sin^{k-1}\psi d\psi\\
  &=-\int_{-\pi/2}^{\pi/2}f_t^2\sin^{k-1}\psi d\psi\le0
\end{align}
with $E_t(f)=0$ iff $f_t=0$. On the other hand, $E(f)\ge0$ so for
generic initial data we would expect energy to asymptotically decrease
to one of its local minimas. Such minimas can be located only at
stationary points which are not unstable under linear
perturbation. Due to the existence of conformal Killing field
$K=\sin(\psi)\frac{\partial}{\partial \psi}$ (with $\mathcal{L}_K
g=-2\cos\psi g$ TODO:check) the only local minimas are constant
solutions $f_0=k\pi$.

%%% Local Variables:
%%% mode: latex
%%% TeX-master: "master"
%%% End:

\section{Gradient flow}
\label{sec:gradient-flow}

We can approach the problem of existence and the properties of
critical points of a functional in several ways. For example in one
dimensional case one can prove the existence of solutions to
Euler-Lagrange equations using some ODE techniques, but this rarely
gives insight into the geometry underlying the functional. Another
more sophisticated way is to analyse the level sets of the functional
and use the infinite dimensional variant of the Morse theory to obtain
critical points as it has been done e.g. in
\cite{Corlette2001}. Finally, there is a powerful technique called the
gradient flow, which consists of defining the flow, which solutions
move along the gradient lines of the functional in the direction of
the negative gradient. \\

The intuition behind the gradient flow is that, if it exists for all
times, it will push the solution into the minimum of the functional
(if the functional is bounded from below). This approach is especially
useful for finding the harmonic map of a given degree, because the
degree is conserved during the flow.

The formal setup for the gradient flow of the Dirichlet energy
\eqref{eq:6} is
\begin{align}
  \label{eq:32}
    \begin{split}
    F\big|_{t=0}&=F_0\\
    \partial_t F^A&=-\delta E(F^A)=\tau(F)^A
  \end{split}
\end{align}
where $\tau(F)$ is
\begin{align}
  \label{eq:33}
  \tau(F)^C=\Delta_g F^C+(\Gamma_{h(F)})_{AB}^{C}\frac{\partial
    F^A}{\partial x^a}\frac{\partial F^B}{\partial x^b}g^{ab}
\end{align}
and generally is a quasi-linear operator. By our assumption on the
domain and the target, \eqref{eq:32} is the set of parabolic
PDE's. Utilizing \eqref{eq:32} we get
\begin{align}
  \label{eq:34}
  \frac{dE}{dt}=-\int_M \partial_t F_A\partial_t F_B h^{AB}
  dV_M\le0
\end{align}
with $dE/dt=0$ if and only if $F$ is harmonic. The flow \eqref{eq:32}
therefore reduces the energy and asymptotically tends to a
critical point of $E$.\\

The question if there exists the harmonic map of the same homotopy
class as the initial data $F_0$ is thus reduced to the existence of
the solution to \eqref{eq:32} for all $t$ which is not singular as
$t\ra\infty$. This method was used in \cite{Eells1964} to prove the
theorem \ref{thm:Eells-Sampson}, which now can be reformulated in the
following form:

\begin{theorem}[Eells-Sampson \cite{Eells1964}]\label{thm:Eells-Sampson2}
  If $N$ is compact and has non-positive sectional Riemannian
  curvature, then for every $F_0$ the solution to \eqref{eq:32} exists
  and converges to the minimizing harmonic map as $t\ra\infty$.
\end{theorem}

Unfortunately the technique of heat flow is not always so successful
-- the solutions to \eqref{eq:32} are not guaranteed to exists for
arbitrary large times. Actually, there is criterion which guarantees
that the solution will cease to exist at some time $T$ which is given
by Struwe.

\begin{theorem}[Struwe \cite{Struwe1996}]\label{thm:Struwe}
  For any time $T>0$ there exists a constant $\epsilon=\epsilon(T)>0$
  such that for any map $F_0:M\ra N$ which is not homotopic to a
  constant and satisfies $E(F_0)<\epsilon$ the solution $F$ to
  \eqref{eq:32} must blow up before time $2T$.
\end{theorem}

In the case when $M$ has non-positive Riemannian curvature the theorem
is still true but empty, because within a given homotopy class, the
smooth harmonic map attains the global minimum which is greater than
the constant $\epsilon(T)$, thus no initial data can fulfill the
criterion $E(F_0)<\epsilon$.

\subsection{Gradient flow for $S^k\ra S^k$}
\label{sec:gradient-flow-skra}

From this point we will assume the map $F:S^k\ra S^k$ is consistent
with the ansatz \eqref{eq:17} and we shall consider the functional
$E(f)$ as defined in \eqref{eq:20} as a base for
\eqref{eq:32}. Applying the gradient flow to \eqref{eq:20} we obtain
the following initial boundary problem
\begin{equation}
  \label{eq:en_flow}
  \begin{split}
    f_t&=\frac{1}{\sin^{k-1}\psi}\left(\sin^{k-1}\psi
      f'\right)'-\frac{(k-1)}{2}\frac{\sin2f}{\sin^2\psi},\\
    f(0,\psi)&=g(\psi),\\
    f(t,0)&=0\quad f(t,\pi)=m\pi.
  \end{split}
\end{equation}
For convenience we have set $g(0)=0$ and $g(\pi)=m\pi$. The choice of
$m$ is responsible for the homotopy degree of $g$ and therefore
$f$. Under evolution according to \eqref{eq:34} the energy decreases
\begin{align}
  \label{eq:35}
  \frac{dE}{dt}=-\int_0^{\pi}f_t^2\sin^{k-1}\psi d\psi<0
\end{align}
unless $f$ is a stationary point of $E(f)$. By the theorem
\ref{thm:harmonic-map-index} the only point to which the flow can
converge starting from a generic initial data is the constant map
$f=n\pi$ for which the energy is zero. The flow will thus reduce the
energy from $E(g)$ to an arbitrary small value. If the initial data is
not homotopic to a constant, the theorem \ref{thm:Struwe} comes into
play, which combined with theorem \ref{thm:harmonic-map-index}
guarantees that the blow up will occur. Moreover, by the numerical
evidence, even if the boundary values are of the form $g(0)=g(\pi)=0$
the blow up can occur
if the initial data lies in the attraction pool of e.g. $f=\pi$.\\

The flow \eqref{eq:en_flow} conserves the parity of $g$, namely if
\begin{align}
  \label{eq:36}
  g(\psi)=\pm g(\pi-\psi)
\end{align}
then at any given time $t\ge0$ we have
\begin{align}\label{eq:37}
  f(t,\psi)=\pm f(t,\pi-\psi).
\end{align}
Our aim is to make the flow converge to one of the stationary points
of $E$, of which we know that have parity specified by
\eqref{eq:28}. We will henceforth denote $g_+$ and $g_-$ as initial
data obeying
\begin{align}
  \label{eq:38}
  g_+(\psi)=g_+(\pi-\psi),\\
  g_-(\psi)=-g_-(\pi-\psi)
\end{align}
which evolve in the apace of symmetric and antisymmetric maps
respectively. We shall call such subspaces $H_+$ and $H_-$
respectively. For each such subspace there are global energy minima
which are $f_0$ and $f_1$ respectively. The energy minimizing property
of $f_0$ is a straight forward, the case of $f_1$ being the energy
minimizer for $H_-$ is not that obvious but follows from the Morse
analysis made in \cite{Corlette2001}. The generic initial data with
suitable boundary conditions in any of those subspaces will converge
to one of their respective vacua unless the blow up occurs. We will
discuss the linear stability of the vacua $f_0$ and $f_1$ in the
following section.

%%% Local Variables:
%%% mode: latex
%%% TeX-master: "master"
%%% End:

\section{Gradient flow}
\label{sec:gradient-flow}

One can approach determining the properties of the critical points of
a functional by numerous ways. For example one can prove the existence
of solutions to Euler-Lagrange equations using some ODE techniques,
but this rarely gives any insight into the geometry underlying the
functional. Another more sophisticated way is to analyse the level
sets of the functional and use the infinite dimensional variant of the
Morse theory to obtain the number of critical points. There is however
a distinguishable technique called the gradient flow, which consists
of defining the flow, which solutions move along the gradient lines of
the functional in the direction of the negative gradient.\\


The intuition behind the gradient flow is that if we start with some
initial state, the gradient flow will push the solution into the
minimum of the functional (if the functional is bounded from below),
and if the flow exists for all times it will always obtain its
asymptotic state. This approach is especially useful for finding the
harmonic map of a given degree, because the degree is conserved during
the flow, so the flow can possibly deform any map into its homotopy
equivalent harmonic map. However, this scenario can hardly ever be
realized because the flow usually runs into singular solutions in
finite or infinite time. This is the case for Dirichlet
energy of the maps $S^k\ra S^k$.\\

The formal setup for the gradient flow of the Dirichlet energy
\eqref{eq:6} is
\begin{align}
  \label{eq:29}
    \begin{split}
    F\big|_{t=0}&=F_0\\
    \partial_t F^A&=-\frac{\delta E}{\delta F_A}=-L(F)^A
  \end{split}
\end{align}
Where $L$ is calculated to be
\begin{align}
  \label{eq:30}
  -L(F)^C=\Delta_g F^C+(\Gamma_{h(F)})_{AB}^{C}\frac{\partial
    F^A}{\partial x^a}\frac{\partial F^B}{\partial x^b}g^{ab}
\end{align}
and generally is a quasi-linear operator. By our assumption on the
domain and the target, \eqref{eq:29} is the set of parabolic
PDE's. The differentiation of \eqref{eq:29} w.r.t. $t$ gives
\begin{align}
  \label{eq:31}
  E_t(F)=-\int_M \partial_t F_A\partial_t F_B h^{AB} dV_M\le0
\end{align}
with $E_t(F)=0$ if and only if $F$ is harmonic. The flow \eqref{eq:29}
therefore reduces the energy and asymptotically tends to some
critical point of $E$.\\

The question if there exists the harmonic map of the same homotopy
class as the initial data $F_0$ is thus reduced to the existence of
the solution to \eqref{eq:29} for all $t$ which is not singular at
$t\ra\infty$. This method was used in \cite{Eells1964} to prove the
theorem \ref{thm:Eells-Sampson}, which now can be reformulated in the
following form.

\begin{theorem}[Eells-Sampson \cite{Eells1964}]\label{thm:Eells-Sampson2}
  If $N$ is compact and has non-positive Riemannian curvature, then
  for every $F_0$ the solution to \eqref{eq:29} exists for
  $t\in[0,\infty)$ and converges to a harmonic map for $t\ra\infty$.
\end{theorem}

As it has been mentioned, the solutions to \eqref{eq:29} are not
guaranteed to exists for arbitrary large times. The criterion which
guarantees that the solution will blow up at some time $T$ is given by
Struwe

\begin{theorem}[Struwe]\label{thm:Struwe}
  For any time $T>0$ there exists a constant
  $\epsilon=\epsilon(M,N,T)>0$ such that for any map $F_0:M\ra N$
  which is not homotopic to a constant and satisfies $E(F_0)<\epsilon$
  the solution $F$ to \eqref{eq:29} must blow up before time $2T$.
\end{theorem}

In the case when $M$ has non-positive Riemannian curvature the theorem
also applies but, within a given homotopy class, the smooth harmonic
map obtains the global minimum which is greater than the constant
$\epsilon(M,N,T)$, thus no initial data can fulfill the criterion
$E(F_0)<\epsilon$.

\subsection{Gradient flow for $S^k\ra S^k$}
\label{sec:gradient-flow-skra}

From this point we will assume the map $F:S^k\ra S^k$ is consistent
with the ansatz \eqref{eq:16} and we shall consider the functional
$E(f)$ as defined in \eqref{eq:21} as a base for
\eqref{eq:29}. Applying the gradient flow to \eqref{eq:21} we obtain
the following Cauchy problem
\begin{equation}
  \label{eq:en_flow}
  \begin{split}
    f_t&=\frac{1}{\sin^{k-1}\psi}\left(\sin^{k-1}\psi
      f'\right)'-\frac{(k-1)}{2}\frac{\sin2f}{\sin^2\psi},\\
    f(0,\psi)&=g(\psi),\\
    f(t,0)&=g(0)\quad f(t,\pi)=g(\pi).
  \end{split}
\end{equation}
For convenience we have set $g(0)=0$ and $g(\pi)=m\pi$. The choice of
$m$ is responsible for the homotopy degree of $g$ and therefore
$f$. Under evolution according to \eqref{eq:31} the energy decreases
\begin{align}
  \label{eq:32}
  E_t(f)=-\int_0^{\pi}f_t^2\sin^{k-1}\psi d\psi<0
\end{align}
unless $f$ is a stationary point of $E(f)$. By the theorem
\ref{thm:harmonic-map-index} the only point to which the flow can
converge starting from a generic initial data is the constant map
$f=n\pi$ for which The energy is zero. The flow will thus reduce the
energy to an arbitrary small value. If the initial data is not
homotopic to a constant, the theorem \ref{thm:Struwe} comes into play,
which combined with \ref{thm:harmonic-map-index} guarantees that the
blow up will occur. Moreover, by the numerical evidence, even if the
boundary values are of the form $g(0)=g(\pi)=0$ the blow up can occur
if the initial data lies in the attraction pool of e.g. $f=\pi$.\\

The flow \eqref{eq:en_flow} conserves the parity of $g$, namely if
\begin{align}
  \label{eq:33}
  g(\psi)=\pm g(\pi-\psi)
\end{align}
then at any given time $t\ge0$ we have
\begin{align}\label{eq:35}
  f(t,\psi)=\pm f(t,\pi-\psi).
\end{align}
Our aim is to make the flow converge to one of the stationary points
of $E$, which we know that have parity specified by \eqref{eq:34}. We
will from now on use $g_+$ and $g_-$ as initial data obeying
\begin{align}
  \label{eq:36}
  g_+(\psi)=g_+(\pi-\psi),\\
  g_-(\psi)=-g_-(\pi-\psi)
\end{align}
which evolve inside the symmetric and antisymmetric subspaces
respectively. We shall call such subspaces $H_+$ and $H_-$
respectively. On each such subspace there are global energy minima
which are $f_0$ and $f_1$ respectively. The energy minimizing property
of $f_0$ is a straight forward, the case of $f_1$ being the minimizer
for $H_-$ is not that obvious but follows from the Morse analysis made
in \cite{Corlette2001}. The generic initial data in any of those
subspaces will converge to one of their respective vacua unless the
blow up occurs. We will discuss the linear stability of the vacua
$f_0$ and $f_1$ in the following section.

%%% Local Variables:
%%% mode: latex
%%% TeX-master: "master"
%%% End:

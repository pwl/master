\section{Gradient flow}
\label{sec:gradient-flow}

We can approach the problem of existence and the properties of
critical points of a functional in several ways. For example in one
dimensional case one can prove the existence of solutions to
Euler-Lagrange equations using some ODE techniques, but this rarely
gives insight into the geometry underlying the functional. Another
more sophisticated way is to analyse the level sets of the functional
and use the infinite dimensional variant of the Morse theory to obtain
critical points as it has been done e.g. in
\cite{Corlette2001}. Finally, there is a powerful technique called the
gradient flow, which consists of defining the flow, which moves along
the gradient lines of the functional in the direction of
the negative gradient. \\

The intuition behind the gradient flow is that, if it exists for all
times, it will push the solution into a minimum of the functional (if
the functional is bounded from below). This approach is especially
useful in finding a harmonic map of a given degree, because the degree
is conserved during the flow.

The formal setup for the gradient flow of the Dirichlet energy
\eqref{eq:7} is
\begin{align}
  \label{eq:33}
    \begin{split}
    F\big|_{t=0}&=F_0,\\
    \partial_t F^A&=-\delta E(F^A)=\tau(F)^A
  \end{split}
\end{align}
where $\tau(F)$ is
\begin{align}
  \label{eq:34}
  \tau(F)^C=\Delta_g F^C+(\Gamma_{h(F)})_{AB}^{C}\frac{\partial
    F^A}{\partial x^a}\frac{\partial F^B}{\partial x^b}g^{ab}.
\end{align}
By our assumption on the domain and the target, \eqref{eq:33} is the
set of quasi-linear parabolic PDE's. Using \eqref{eq:33} we get
\begin{align}
  \label{eq:35}
  \frac{dE}{dt}=-\int_M \partial_t F_A\partial_t F_B h^{AB}
  dV_M\le0
\end{align}
with $dE/dt=0$ if and only if $F$ is harmonic. The flow \eqref{eq:33}
therefore reduces the energy and asymptotically tends to a
critical point of $E$.\\

The question if there exists a harmonic map in the same homotopy class
as the initial data $F_0$ is thus reduced to the question of existence
of a solution of \eqref{eq:33} for all $t$ which is not singular as
$t\ra\infty$. This method was used in \cite{Eells1964} to prove
Theorem \ref{thm:Eells-Sampson}, which now can be reformulated in the
following form:

\begin{theorem}[Eells-Sampson \cite{Eells1964}]\label{thm:Eells-Sampson2}
  If $N$ is compact and has non-positive sectional Riemannian
  curvature, then for every $F_0$ the solution to \eqref{eq:33} exists
  for all times and converges to the minimizing harmonic map as
  $t\ra\infty$.
\end{theorem}

Unfortunately the technique of heat flow is not always so successful
-- the solutions to \eqref{eq:33} are not guaranteed to exist for
arbitrary large times. Actually, there is criterion which guarantees
that the solution will cease to exist at some time $T$:

\begin{theorem}[Struwe \cite{Struwe1996}]\label{thm:Struwe}
  For any time $T>0$ there exists a constant $\epsilon=\epsilon(T)>0$
  such that for any map $F_0:M\ra N$ which is not homotopic to a
  constant and satisfies $E(F_0)<\epsilon$ the solution $F$ to
  \eqref{eq:33} must blow up before time $2T$.
\end{theorem}

In the case when $M$ has non-positive Riemannian curvature the theorem
is still true but empty, because within a given homotopy class, the
smooth harmonic map attains the global minimum which is greater than
the constant $\epsilon(T)$, thus no initial data can fulfill the
criterion $E(F_0)<\epsilon$.

\subsection{Gradient flow for $S^k\ra S^k$}
\label{sec:gradient-flow-skra}

From this point we will assume that the map $F:S^k\ra S^k$ satisfies
the ansatz \eqref{eq:18} and we shall consider the functional $E(f)$
as defined in \eqref{eq:21} as a base for \eqref{eq:33}. Applying the
gradient flow to \eqref{eq:21} we obtain the following initial
boundary problem

\begin{equation}
  \label{eq:en_flow}
  \begin{split}
    f_t&=\frac{1}{\sin^{k-1}\psi}\left(\sin^{k-1}\psi
      f'\right)'-\frac{(k-1)}{2}\frac{\sin2f}{\sin^2\psi},\\
    f(0,\psi)&=g(\psi),\\
    f(t,0)&=g(0)=0\quad f(t,\pi)=g(\pi)=m\pi.
  \end{split}
\end{equation}

The integer $m$ is equal to the topological degree of the map.  By
\eqref{eq:35} the energy decreases

\begin{align}
  \label{eq:36}
  \frac{dE}{dt}=-\int_0^{\pi}f_t^2\sin^{k-1}\psi d\psi<0
\end{align}

unless $f$ is a stationary point of $E(f)$. By Theorem
\ref{thm:harmonic-map-index} the only point to which the flow can
converge starting from generic initial data is the constant map
$f=n\pi$ for which the energy is zero. The flow will thus reduce the
energy from $E(g)$ to an arbitrary small value. If the initial data is
not homotopic to a constant, Theorem \ref{thm:Struwe} comes into play,
which combined with theorem \ref{thm:harmonic-map-index} guarantees
that the blow-up will occur. Moreover, by the numerical evidence, even
if the boundary values are of the form $g(0)=g(\pi)=0$ the blow-up can
occur if initial data lie in the basin of attraction of $f=\pi$.\\

The flow \eqref{eq:en_flow} conserves the parity of $g$, namely if
\begin{align}
  \label{eq:37}
  g(\psi)=\pm g(\pi-\psi)
\end{align}
then at any given time $t\ge0$ we have
\begin{align}\label{eq:38}
  f(t,\psi)=\pm f(t,\pi-\psi).
\end{align}
Our aim is to make the flow converge to one of the stationary points
of $E$, of which we know that have parity specified by
\eqref{eq:29}. We will henceforth denote $g_+$ and $g_-$ as initial
data obeying
\begin{align}
  \label{eq:39}
  g_+(\psi)&=g_+(\pi-\psi),\\
  g_-(\psi)&=-g_-(\pi-\psi)
\end{align}
which evolve in the apace of symmetric and antisymmetric maps
respectively. We shall call such subspaces $H_+$ and $H_-$ and define
them

\begin{align}
  \label{eq:40}
  H_\pm=\{f(t,\psi)|\quad f(t,\psi)=\pm f(t,\pi-\psi)\}.
\end{align}

We also remark that

\begin{align}
  \label{eq:41}
  f_{2n}\in H_+,\quad f_{2n+1}\in H_-.
\end{align}

For each such subspace there are global energy minima which are $f_0$
and $f_1$ respectively. The energy minimizing property of $f_0$ is a
straightforward, the case of $f_1$ being the energy minimizer for
$H_-$ is not that obvious but follows from the Morse analysis made in
\cite{Corlette2001}. Generic initial data with suitable boundary
conditions in any of those subspaces will converge to one of their
respective ground states unless the blow up occurs. We will discuss
the linear stability of the ground states $f_0$ and $f_1$ in the
following section.

%%% Local Variables:
%%% mode: latex
%%% TeX-master: "master"
%%% End:

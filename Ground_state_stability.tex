\section{Stability of harmonic maps}
\label{sec:stab-ground-stat}

As we know already, $f_0$ is the global minimum of the Dirichlet
energy and we expect it to be linearly stable. Also $f_1$ should turn
out to be linearly stable under antisymmetric perturbations, and
unstable under the symmetric ones due to theorem
\ref{thm:harmonic-map-index} and \eqref{eq:37}. The linear stability
can be approached by determining the eigenvalues of the Hessian
$\delta^2E(f_n)$ introduced in \eqref{eq:29} which corresponds to
solving the eigenproblem

\begin{align}
  \label{eq:39}
  \mathcal{L}_n v = \lambda v. \\
  \label{eq:40}
  \mathcal{L}_n v = -\left(v''+(k-1)\cot\psi
    v'-(k-1)\frac{\cos(2f_n)}{\sin^2\psi}v\right)
\end{align}

with Dirichlet boundary conditions for $v$. If we perturb $f_n$ in the
direction of $v$, the perturbed state will evolve as

\begin{align}
  \label{eq:41}
  f=f_n+e^{-\lambda t}v
\end{align}
which follows from linearizing \eqref{eq:en_flow} around $f_n$.

By $f_n(0)=0$ and $f_n(\pi)=0$ or $f_n(\pi)=1$, the critical exponents
at $\psi=0$ and $\psi=\pi$ of solutions to \eqref{eq:46} are $1$ and
$1-k$, of which only the first one corresponds to the solution
fulfilling the Dirichlet condition. We will denote the $m$-th
eigenvalue of $\mathcal{L}_n$ as $\lambda_m^n$ along with its
respective eigenvectors $v_m^n$ with $m\in\mathbb{N}_0$. The
eigenvalues to any $n$ are ordered as follows
\begin{align}
  \label{eq:75}
  \lambda_n^0<\lambda_n^1<\lambda_n^2<\dots.
\end{align}

Since near both ends of the interval $(0,\pi)$ only one solution is
allowed the operator $\mathcal{L}_n$ does not have degenerate
eigenspaces. Along with the fact that $\mathcal{L}_n$ is symmetric
under $\psi\ra\pi-\psi$ (by \eqref{eq:28}), it follows that $v$ can be
either symmetric or antisymmetric. We can use the Sturm-Liouville
theorem to determine the exact parity of eigenvectors by the fact that
$v_m^n$ should have $m-1$ zeroes, from which it follows that
\begin{align}
  \label{eq:44}
  v_m^n(\psi)=(-1)^m v_m^n(\pi-\psi).
\end{align}

As mentioned in the proof of theorem \ref{thm:harmonic-map-index}, for
$n\ge1$ each $f_n$ has the eigenvector generated by the conformal
Killing field and defined as follows
\begin{align}
  \label{eq:42}
  v_{\text{conf}}^n=\sin\psi f_n'(\psi),
\end{align}
corresponding to the $\lambda_{\text{conf}}^n=(2-k)$. As $f_n'$ has
exactly $n-1$ zeroes, there are $n-1$ eigenvalues smaller than
$\lambda_{\text{conf}}^n=(2-k)$ which follows from the Sturm-Liouville
theorem. We shall therefore denote
\begin{align}
  \label{eq:43}
  v_{n-1}^n=\sin\psi f_n'(\psi)\\
  \lambda_{n-1}^n=(2-k).
\end{align}

\subsection{Stability of $f_0$}
\label{sec:stability-f_0}

By \eqref{eq:39}\eqref{eq:40}, the eigenproblem for $f_0$ is
\begin{align}
  \label{eq:45}
  \mathcal{L}_0 v = \lambda v. \\
  \label{eq:46}
  \mathcal{L}_0 v = -v''-(k-1)\cot\psi v'+\frac{(k-1)}{\sin^2\psi}v
\end{align}
We can multiply \eqref{eq:39} by $v\sin^{k-1}\psi$ and integrate on
the interval $[0,\pi]$ to get
\begin{align}\label{eq:47}
  \lambda\int_0^\pi v^2w
  d\psi=\int_0^\pi\left(v'^2+\frac{k-1}{\sin^2\psi}v^2\right)\sin^{k-1}d\psi<0.
\end{align}

All the terms under the integrals are positive for $v\ne0$ so
$\lambda>0$ and $f_0$ is linearly stable. We can however calculate the
spectrum of \eqref{eq:46} explicitly. First we change the independent
function to
\begin{align}
  \label{eq:48}
  w(\psi)= v(\psi)\sin^{(k-1)/2}\psi
\end{align}
and the eigenproblem \eqref{eq:45} simplifies to
\begin{align}
  \label{eq:49}
  -w''+V_0(\psi)w=\left(\lambda-\frac{1}{2}(k-1)\right)w\\
  V_0(\psi)=\frac{1}{4}(k^2-1)\cot^2\psi.
\end{align}
The general solution to \eqref{eq:49} can be written in terms of the
associated Legendre polynomials
\begin{align}
  \label{eq:50}
  w(\psi)=\sqrt{\sin\psi}\left(
    AP^\alpha_{\beta}(\cos(\psi))+BP^{-\alpha}_{\beta}(\cos(\psi))
  \right)\\
  \label{eq:51}
  \alpha=\frac{k}{2}\quad \beta=-\frac{1}{2}+\frac{1}{2}\sqrt{(k-1)^2+4\lambda}.
\end{align}
As $\sqrt{\sin\psi}P_\beta^{\alpha}(\cos\psi)\sim\psi^{1/2-\alpha}$
grows too rapidly at $\psi=0$, we discard the first solution by
setting $A=0$. We now analyse the behaviour at $\psi=\pi$ by expanding
$P_\beta^{-\alpha}$ around $\cos\pi=-1$
\begin{align}
  \label{eq:52}
  &P_\beta^{-\alpha}(z)=\\
  &\frac{2^{\alpha/2}\Gamma(\alpha)}{\Gamma(\alpha-\beta)\Gamma(\beta+\alpha+1)}
  (z+1)^{-\alpha/2}(1+\mathcal{O}((z+1)))\\
  &-\frac{1}{\pi}2^{-\alpha/2}\sin(\pi\beta)\Gamma(-\alpha)
  (1+z)^{\alpha/2}(1+\mathcal{O}((z+1)))
\end{align}
The quantization of $\lambda$ is obtained by setting the coefficient
in the first term to zero. This can be achieved only by selecting
$\beta$ so that one of the functions $1/\Gamma(z)$ will zero, which
induces $\beta\in\R$. But then $\beta+\alpha+1>0$ and the only
possible zeroes are for
\begin{align}
  \label{eq:53}
  \alpha-\beta=-m\\
  m\in\mathbb{N}_0
\end{align}
which gives rise to the following quantization of $\lambda$
\begin{align}
  \label{eq:54}
  \lambda_m^0&=(1+m)(k+m),
\end{align}
which yields positive eigenvalues for any $k\ge1$. The eigenvectors
corresponding to \eqref{eq:20} are
\begin{align}
  \label{eq:55}
  w_m^0(\psi)=\sqrt{\sin\psi}P_{k/2+m}^{-k/2}(\cos\psi).
\end{align}
By the parity properties of $P_{k/2+m}^{-k/2}$ we get the symmetric
and antisymmetric families of eigenvectors
\begin{align}
  \label{eq:56}
  w_m^0(\psi)=(-1)^m w_m^0(\pi-\psi)
\end{align}
which agrees with \eqref{eq:44}. Going back to the eigenvectors of
$\mathcal{L}_0$ we obtain
\begin{align}
  \label{eq:57}
  v_m(\psi)=\sin(\psi)C^{\left(\frac{1+k}{2}\right)}_m(\cos\psi)
\end{align}
where $C^{(\lambda)}_m$ are the Gegenbauer polynomials.

\subsection{Stability of $f_1$}
\label{sec:stability-f_1}

For $f_1$ the eigenproblem is stated as
\begin{align}
  \label{eq:58}
  \mathcal{L}_1 v = \lambda v. \\
  \label{eq:59}
  \mathcal{L}_1 v = -v''-(k-1)\cot\psi v'+(k-1)\frac{\cos2\psi}{\sin^2\psi}v
\end{align}

We can get the bound similar to \eqref{eq:47} but valid only for the
antisymmetric eigenvectors (otherwise the boundary terms from the
integration by parts do not vanish) by integrating \eqref{eq:58} on
the interval $[0,\pi/2]$
\begin{align}
  \label{eq:f1_lambda}
  \lambda\int_0^{\pi/2} v^2w
  d\psi=-\int_0^{\pi/2}\left(v'^2+(k-1)\frac{\cos2\psi}{\sin^2\psi}v^2\right)\sin^{k-1}d\psi<0
\end{align}
\eqref{eq:58} is also solvable by the procedure similar to that used
in the case of $f_0$ which yields
\begin{align}
  \label{eq:60}
  v_m^1(\psi)=\sin(\psi)C^{\left(\frac{1+k}{2}\right)}_m(\cos\psi), \\
  \label{eq:61}
  \lambda_m^1=2 + k (-1 + m) + m + m^2.
\end{align}
The only eigenvalue smaller than zero is as expected $\lambda_0^1=2-k$
and, by $C^{(a)}_0(x)=1$, the corresponding eigenvector is
$v_0^1=\sin\psi$.

\subsection{Stability of $f_2$ and $f_3$}
\label{sec:stability-f_2-f_3}

Although we cannot solve \eqref{eq:39} analytically for $n\ge2$ we
know the number of negative eigenvalues from theorem
\ref{thm:harmonic-map-index} and the parity of the corresponding
eigenvectors from \eqref{eq:44}. We utilize that knowledge to cut down
the number of unstable directions by restricting the possible
perturbations to those belonging to $H_+$ or $H_-$. By this procedure
we kill all the antisymmetric modes of $\mathcal{L}_2$ and the
symmetric ones of $\mathcal{L}_3$ reducing the number of instabilities
of $f_2$ and $f_3$ to one. We can thus write the perturbed solutions
around $f_2$ and $f_3$ as
\begin{align}
  \label{eq:62}
  f_+=f_2+A_0e^{-\lambda_2^0t}v_2^0+A_2e^{-\lambda_2^2t}v_2^2+\dots,\\
  f_-=f_3+B_1e^{-\lambda_3^1t}v_3^1+B_3e^{-\lambda_3^3t}v_3^3+\dots.
\end{align}
The first modes are the unstable while the latter are stable. This
result does not help much at the moment but it will be used in the
following sections where we shall determine the first two modes of
$f_{2,3}$ by solving the PDE \eqref{eq:en_flow} numerically.


%%% Local Variables:
%%% mode: latex
%%% TeX-master: "master"
%%% End:

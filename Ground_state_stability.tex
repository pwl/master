\section{Stability of harmonic maps}
\label{sec:stab-ground-stat}

As we know already, $f_0$ is the global minimum of the Dirichlet
energy and we expect it to be linearly stable. Also $f_1$ should turn
out to be linearly stable under antisymmetric perturbations, and
unstable under the symmetric ones due to theorem
\ref{thm:harmonic-map-index} and \eqref{eq:38}. The linear stability
can be approached by determining the eigenvalues of the Hessian
$\delta^2E(f_n)$ introduced in \eqref{eq:30} which corresponds to
solving the eigenproblem

\begin{align}
  \label{eq:42}
  &\mathcal{L}_n v = \lambda v, \\
  \label{eq:43}
  &\mathcal{L}_n v = -\frac{1}{\sin^{k-1}\psi}\left(\sin^{k-1}\psi
    v'\right)'+(k-1)\frac{\cos2f_n}{\sin^2\psi}v.
\end{align}
The Hilbert space for this problem is
$\mathcal{H}=L^2([0,\pi],\sin^{k-1}\psi d\psi)$. If we perturb $f_n$
in the direction of $v$, the perturbed state will evolve as

\begin{align}
  \label{eq:44}
  f=f_n+e^{-\lambda t}v
\end{align}

which follows from linearizing \eqref{eq:en_flow} around $f_n$.\\

By $f_n(0)=0$, the equation \eqref{eq:42} near $\psi=0$ takes the form
\begin{align}
  \label{eq:45}
  v''+\frac{k-1}{\psi}v'-\frac{k-1}{\psi^2}v=0.
\end{align}
We obtain the indicial equation by substituting $v(\psi)=\psi^\gamma$
obtaining
\begin{align}
  \label{eq:46}
  \gamma(\gamma-1)+(k-1)\gamma-(k-1)=0
\end{align}
with the solutions
\begin{align}
  \label{eq:47}
  \gamma=1\quad\text{or}\quad\gamma=1-k,
\end{align}
of which only the first one corresponds to $v\in\mathcal{H}$. The
analogous asymptotic behaviour can be derived at $\psi=\pi$ where also
only one solution is in $\mathcal{H}$. We will denote the $m$-th
eigenvalue of $\mathcal{L}_n$ as $\lb{n}{m}$ and the associated
eigenvectors as $\vn{m}{n}$ with $m\in\mathbb{N}_0$. The eigenvalues
are ordered as follows
\begin{align}
  \label{eq:48}
  \lb{n}{0}<\lb{n}{1}<\lb{n}{2}<\dots\quad.
\end{align}
From the fact that $\mathcal{L}_n$ is symmetric under
$\psi\ra\pi-\psi$ (by \eqref{eq:29}), it follows that $v$ can be
either symmetric or antisymmetric. We can use the Sturm oscillation
theorem \cite{Reed4} to determine the exact parity of eigenvectors by
the fact that $\vn{m}{n}$ should have $m-1$ zeroes, from which it
follows that
\begin{align}
  \label{eq:49}
  \vn{m}{n}(\psi)=(-1)^m \vn{m}{n}(\pi-\psi).
\end{align}

As mentioned in the proof of theorem \ref{thm:harmonic-map-index}, for
$n\ge1$ each $f_n$ has the eigenvector generated by the conformal
Killing field and defined as follows
\begin{align}
  \label{eq:50}
  \vn{\text{conf}}{n}=\sin\psi f_n'(\psi),
\end{align}
corresponding to the $\lambda_{\text{conf}}^n=(2-k)$. As $f_n'(\psi)$
has exactly $n-1$ zeroes, there are $n-1$ eigenvalues smaller than
$\lambda_{\text{conf}}^n=(2-k)$ which follows from the Sturm
oscillation theorem. We shall therefore denote
\begin{align}
  \label{eq:51}
  \vn{n-1}{n}=\sin\psi f_n'(\psi)\quad
  \lb{n}{n-1}=(2-k).
\end{align}

\subsection{Stability of $f_0$}
\label{sec:stability-f_0}

By \eqref{eq:42}\eqref{eq:43}, the eigenproblem for $f_0$ is
\begin{align}
  \label{eq:52}
  &\mathcal{L}_0 v = \lambda v, \\
  \label{eq:53}
  &\mathcal{L}_0 v = -\frac{1}{\sin^{k-1}\psi}\left(\sin^{k-1}\psi
    v'\right)'+\frac{(k-1)}{\sin^2\psi}v.
\end{align}
We can multiply \eqref{eq:42} by $v\sin^{k-1}\psi$ and integrate on
the interval $[0,\pi]$ to get
\begin{align}\label{eq:54}
  \lambda\int_0^\pi v^2w
  d\psi=\int_0^\pi\left(v'^2+\frac{k-1}{\sin^2\psi}v^2\right)\sin^{k-1}d\psi>0.
\end{align}
All the terms under the integrals are positive for $v\ne0$ so
$\lambda>0$ and $f_0$ is linearly stable. We can calculate the
spectrum of \eqref{eq:53} explicitly. First we change the independent
function to
\begin{align}
  \label{eq:55}
  w(\psi)= v(\psi)\sin^{(k-1)/2}\psi
\end{align}
and the eigenproblem \eqref{eq:52} simplifies to
\begin{align}
  \label{eq:56}
  &-w''+V_0(\psi)w=\left(\lambda-\frac{1}{2}(k-1)\right)w,\\
  &V_0(\psi)=\frac{1}{4}(k^2-1)\cot^2\psi.
\end{align}
The solution to \eqref{eq:56} with the proper asymptotic at $\psi=0$
can be written in terms of the associated Legendre polynomials
\begin{align}
  \label{eq:57}
  &w(\psi)=\sqrt{\sin\psi}P^{-\alpha}_{\beta}(\cos(\psi)),\\
  \label{eq:58}
  &\alpha=\frac{k}{2},\quad \beta=-\frac{1}{2}+\frac{1}{2}\sqrt{(k-1)^2+4\lambda}.
\end{align}
We now analyse the behaviour at $\psi=\pi$ by expanding
$P_\beta^{-\alpha}$ around $\cos\pi=-1$
\begin{align}
  \label{eq:59}
  \begin{split}
    &P_\beta^{-\alpha}(z)=\\
    &\frac{2^{\alpha/2}\Gamma(\alpha)}{\Gamma(\alpha-\beta)\Gamma(\beta+\alpha+1)}
    (z+1)^{-\alpha/2}(1+\mathcal{O}((z+1)))\\
    &-\frac{1}{\pi}2^{-\alpha/2}\sin(\pi\beta)\Gamma(-\alpha)
    (1+z)^{\alpha/2}(1+\mathcal{O}((z+1)))
  \end{split}
\end{align}
The quantization of $\lambda$ is obtained by setting the coefficient
in the first term to zero. This can be achieved only by selecting
$\beta$ so that one of the functions $1/\Gamma(z)$ will be zero, which
implies $\beta\in\R$. But then $\beta+\alpha+1>0$ and the only
possible zeroes are for
\begin{align}
  \label{eq:60}
  \alpha-\beta=-m,\quad m\in\mathbb{N}_0
\end{align}
which gives rise to the following quantization of $\lambda$
\begin{align}
  \label{eq:61}
  \lb{0}{m}&=(1+m)(k+m).
\end{align}
The eigenvectors corresponding to \eqref{eq:21} are
\begin{align}
  \label{eq:62}
  \wn{0}{m}(\psi)=\sqrt{\sin\psi}P_{k/2+m}^{-k/2}(\cos\psi).
\end{align}
By the parity properties of $P_{k/2+m}^{-k/2}$ we get the symmetric
and antisymmetric families of eigenvectors
\begin{align}
  \label{eq:63}
  \wn{0}{m}(\psi)=(-1)^m w_m^0(\pi-\psi)
\end{align}
which agrees with \eqref{eq:49}. Going back to the eigenvectors of
$\mathcal{L}_0$ we obtain
\begin{align}
  \label{eq:64}
  \vn{0}{m}(\psi)=\sin(\psi)C^{\left(\frac{1+k}{2}\right)}_m(\cos\psi),
\end{align}
where $C^{(\lambda)}_m$ are the Gegenbauer polynomials connected to
the Legendre polynomials \cite{nist} by
\begin{align}
  \label{eq:103}
  % \begin{split}
  %   P_\beta^{-\alpha}(x)=A(\alpha,\beta)(1-x^2)^{\frac{\alpha}{2}}
  %   C^{(\frac{1}{2}+\alpha)}_{\beta-\alpha}(x),\\
  %   A(\alpha,\beta)=\\
  \mathop{P^{{\mu}}_{{\nu}}\/}\nolimits\!\left(x\right)=\frac{2^{{\mu}}\mathop{\Gamma\/}\nolimits\!\left(1-2\mu\right)\mathop{\Gamma\/}\nolimits\!\left(\nu+\mu+1\right)}{\mathop{\Gamma\/}\nolimits\!\left(\nu-\mu+1\right)\mathop{\Gamma\/}\nolimits\!\left(1-\mu\right)\left(1-x^{2}\right)^{{\mu/2}}}\mathop{C^{{(\frac{1}{2}-\mu)}}_{{\nu+\mu}}\/}\nolimits\!\left(x\right).
  % \end{split}
\end{align}
The relation to Gegenbauer polynomial was emphasized because
$C_m^{(\delta)}(x)$ are polynomials in $x$ of degree $m$. The closed
form of $\vn{m}{0}$ for first few $m$ are given below.
\begin{align}\label{eq:104}
  \begin{split}
    &v_0^{(0)} = \sin (\psi ), \\
    &v_1^{(0)} = \frac{1}{2} (k+1) \sin (2 \psi ), \\
    &v_2^{(0)} = \frac{1}{8} \left(k^2-1\right) \sin (\psi
    )+\frac{1}{8} \left(k^2+4 k+3\right) \sin (3 \psi ), \\
    &v_3^{(0)} = \frac{1}{48} \left(2 k^3+6 k^2-2 k-6\right)
    \sin (2 \psi )+\frac{1}{48} \left(k^3+9 k^2+23 k+15\right) \sin (4
    \psi ).
  \end{split}
\end{align}

\subsection{Stability of $f_1$}
\label{sec:stability-f_1}

For $f_1$ the eigenproblem is stated as
\begin{align}
  \label{eq:65}
  &\mathcal{L}_1 v = \lambda v. \\
  \label{eq:66}
  &\mathcal{L}_1 v = -\frac{1}{\sin^{k-1}\psi}\left(\sin^{k-1}\psi
    v'\right)'+(k-1)\frac{\cos2\psi}{\sin^2\psi}v
\end{align}

We can get the bound similar to \eqref{eq:54} but valid only for the
antisymmetric eigenvectors (otherwise the boundary terms from the
integration by parts do not vanish) by integrating \eqref{eq:65} on
the interval $[0,\pi/2]$
\begin{align}
  \label{eq:f1_lambda}
  \lambda\int_0^{\pi/2} v^2w
  d\psi=\int_0^{\pi/2}\left(v'^2+(k-1)\frac{\cos2\psi}{\sin^2\psi}v^2\right)\sin^{k-1}d\psi>0.
\end{align}
\eqref{eq:65} is also solvable by the procedure similar to that used
in the case of $f_0$ which yields
\begin{align}
  \label{eq:67}
  \vn{1}{m}(\psi)=\sin(\psi)C^{\left(\frac{1+k}{2}\right)}_m(\cos\psi), \\
  \label{eq:68}
  \lb{1}{m}=2 + k (-1 + m) + m + m^2.
\end{align}
The only eigenvalue smaller than zero is $\lb{1}{0}=2-k$ and, by
$C^{(\delta)}_0(x)=1$, the corresponding eigenvector is
$\vn{1}{0}=\sin\psi$. By \eqref{eq:64}, $\vn{m}{0}=\vn{m}{1}$ so the
table \eqref{eq:104} gives the closed forms of $\vn{m}{1}$.

\subsection{Stability of $f_2$ and $f_3$}
\label{sec:stability-f_2-f_3}

Although we cannot solve \eqref{eq:42} analytically for $n\ge2$ we
know the number of negative eigenvalues from theorem
\ref{thm:harmonic-map-index} and the parity of the corresponding
eigenvectors from \eqref{eq:49}. We use that knowledge to reduce the
number of unstable directions by restricting the possible
perturbations to those belonging to $H_+$ or $H_-$. By this procedure
we kill all the antisymmetric modes of $\mathcal{L}_2$ and the
symmetric ones of $\mathcal{L}_3$ reducing the number of instabilities
of $f_2$ and $f_3$ to one. We can thus write the perturbed solutions
around $f_2$ and $f_3$ as
\begin{align}
  \label{eq:69}
  f_+=f_2+A_0e^{-\lb{2}{0}t}\vn{2}{0}+A_2e^{-\lb{2}{2}t}\vn{2}{2}+\dots,\\
  \label{eq:70}
  f_-=f_3+B_1e^{-\lb{3}{1}t}\vn{3}{1}+B_3e^{-\lb{3}{3}t}\vn{3}{3}+\dots.
\end{align}
The first modes are unstable while the latter are stable. This will be
used in the following sections where we shall determine the first two
modes of $f_{2,3}$ by solving the PDE \eqref{eq:en_flow} numerically.


%%% Local Variables:
%%% mode: latex
%%% TeX-master: "master"
%%% End:

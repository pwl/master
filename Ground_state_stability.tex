\section{Stability of harmonic maps}
\label{sec:stab-ground-stat}

In this section we calculate the spectra of perturbation operators
around $f_0$ and $f_1$ in closed form. As we know already, $f_0$ is
the global minimum of the Dirichlet energy and we expect it to be
linearly stable. Also $f_1$ should turn out to be linearly stable
under antisymmetric perturbations, and unstable under the symmetric
ones due to \ref{thm:harmonic-map-index}. The linear stability can be
approached by determining the eigenvalues of the Hessian
$\delta^2E(f_n)$ introduced in \eqref{eq:26} which corresponds to
solving the eigenproblem

\begin{align}
  \label{eq:f0_L}
  \mathcal{L}_n v = \lambda v. \\
  \label{eq:37}
  \mathcal{L}_n v = -\left(v''+(k-1)\cot\psi
    v'-(k-1)\frac{\cos(2f_n)}{\sin^2\psi}v\right)
\end{align}

with Dirichlet boundary conditions for $v$. If we perturb $f_n$ in the
direction of $v$, the evolution of the perturbed state will evolve as

\begin{align}
  \label{eq:42}
  f=f_n+e^{-\lambda t}v
\end{align}
as it follows from linearizing \eqref{eq:en_flow} around $f_n$.

By $f_n(0)=0$ and $f_n(\pi)=m\pi$, the critical exponents at $\psi=0$
and $\psi=\pi$ of solutions to \eqref{eq:37} are $1$ and $1-k$, of
which only the first one fulfills the Dirichlet condition. Since at
both ends of the interval $[0,\pi]$ only one solution is allowed the
operator $\mathcal{L}_n$ does not have degenerate eigenspaces. Along
with the fact that the potential term is symmetric under
$\psi\ra\pi-\psi$ (by \eqref{eq:34}), it follows that $v$
can be either symmetric or antisymmetric.\\

We will denote the $m$-th eigenvalue of $\mathcal{L}_n$ as
$\lambda_m^n$ along with its respective eigenvectors $v_m^n$ with
$m\in\mathbb{N}_0$. As mentioned in the proof of
\ref{thm:harmonic-map-index}, each $f_n$ has one eigenvector generated
by the conformal Killing field and defined as follows
\begin{align}
  \label{eq:43}
  v_{\text{conf}}^n=\sin\psi f_n'(\psi),
\end{align}
corresponding to the $\lambda_{\text{conf}}^n=(2-k)$. As $f_n'$ has
exactly $n-1$ zeroes, there are $n-2$ eigenvalues smaller than
$\lambda_{\text{conf}}^n=(2-k)$ which follows from the Sturm-Liouville
theorem. We shall therefore denote
\begin{align}
  \label{eq:46}
  v_{n-1}^n=\sin\psi f_n'(\psi)\\
  \lambda_{n-1}^n=(2-k).
\end{align}

We can use the Sturm-Liouville theorem also for determining the exact
parity of eigenvectors by the fact that $v_m^n$ should have $m-1$
zeroes. The parity of $v_m^n$ can then be formulated as
\begin{align}
  \label{eq:50}
  v_m^n(\psi)=(-1)^m v_m^n(\pi-\psi).
\end{align}

\subsection{Stability of $f_0$}
\label{sec:stability-f_0}

By \eqref{eq:37}, the eigenproblem for $f_0$ is
\begin{align}
  \label{eq:f0_L}
  \mathcal{L}_0 v = \lambda v. \\
  \label{eq:37}
  \mathcal{L}_0 v = -v''-(k-1)\cot\psi v'+\frac{(k-1)}{\sin^2\psi}v
\end{align}
We can multiply \eqref{eq:f0_L} by $v\sin^{k-1}\psi$ and integrate on
the interval $[0,\pi]$ to get
\begin{align}\label{eq:47}
  \lambda\int_0^\pi v^2w
  d\psi=\int_0^\pi\left(v'^2+\frac{k-1}{\sin^2\psi}v^2\right)\sin^{k-1}d\psi<0.
\end{align}

All the terms under the integrals are positive for $v\ne0$ so
$\lambda>0$ and $f_0$ is linearly stable. We can however calculate the
spectrum of \eqref{eq:37} explicitly. First we change the independent
function to
\begin{align}
  \label{eq:38}
  w(\psi)= v(\psi)\sin^{(k-1)/2}\psi
\end{align}
and the eigenproblem simplifies to
\begin{align}
  \label{eq:39}
  -w''+V_0(\psi)w=\left(\lambda-\frac{1}{2}(k-1)\right)w\\
  V_0(\psi)=\frac{1}{4}(k^2-1)\cot^2\psi.
\end{align}
The general solution to \eqref{eq:39} can be written in terms of the
associated Legendre polynomials
\begin{align}
  \label{eq:40}
  w(\psi)=\sqrt{\sin\psi}\left(
    AP^\alpha_{\beta}(\cos(\psi))+BP^{-\alpha}_{\beta}(\cos(\psi))
  \right)\\
  \label{eq:41}
  \alpha=\frac{k}{2}\quad \beta=-\frac{1}{2}+\frac{1}{2}\sqrt{(k-1)^2+4\lambda}.
\end{align}
As $\sqrt{\sin\psi}P_\beta^{\alpha}(\cos\psi)\sim\psi^{1/2-\alpha}$
grows too rapidly at $\psi=0$, we set $A=0$. We now analyse the
behaviour at $\psi=\pi$ by expanding $P_\beta^{-\alpha}$ around
$\cos\pi=-1$
\begin{align}
  \label{eq:2}
  &P_\beta^{-\alpha}(z)=\\
  &\frac{2^{\alpha/2}\Gamma(\alpha)}{\Gamma(\alpha-\beta)\Gamma(\beta+\alpha+1)}
  (z+1)^{-\alpha/2}(1+\mathcal{O}((z+1)))\\
  &-\frac{1}{\pi}2^{-\alpha/2}\sin(\pi\beta)\Gamma(-\alpha)
  (1+z)^{\alpha/2}(1+\mathcal{O}((z+1)))
\end{align}
We obtain the quantization of $\lambda$ by setting the coefficient in
the first term to zero. This can be achieved only by setting $\beta$
so that one of the functions $1/\Gamma(z)$ will zero, which induces
$\beta\in\R$. But then $\beta+\alpha+1>0$ and the only possible zeroes
are for
\begin{align}
  \label{eq:3}
  \alpha-\beta=-m\\
  m\in\mathbb{N}_0
\end{align}
which gives rise to the following quantization of $\lambda$
\begin{align}
  \label{eq:6}
  \lambda_m^0&=(1+m)(k+m),
\end{align}
which yields positive eigenvalues for any $k\ge1$. The eigenvectors
corresponding to \eqref{eq:21} are
\begin{align}
  \label{eq:11}
  w_m^0(\psi)=\sqrt{\sin\psi}P_{k/2+m}^{-k/2}(\cos\psi).
\end{align}
By the parity properties of $P_{k/2+m}^{-k/2}$ we get the symmetric
and antisymmetric families of eigenvectors
\begin{align}
  \label{eq:44}
  w_m^0(\psi)=(-1)^m w_m^0(\pi-\psi)
\end{align}
which agrees with the parity property. Going back to the eigenvectors
of $\mathcal{L}_0$ we obtain
\begin{align}
  \label{eq:45}
  v_m(\psi)=\sin(\psi)C^{\left(\frac{1+k}{2}\right)}_m(\cos\psi)
\end{align}
where $C^{(\lambda)}_m$ are the Gegenbauer polynomials.

\subsection{Stability of $f_1$}
\label{sec:stability-f_1}

For $f_1$ the eigenproblem is stated as
\begin{align}
  \label{eq:f0_L}
  \mathcal{L}_1 v = \lambda v. \\
  \label{eq:37}
  \mathcal{L}_1 v = -v''-(k-1)\cot\psi v'+(k-1)\frac{\cos2\psi}{\sin^2\psi}v
\end{align}

We can get the bound similar to \eqref{eq:47} but only for
antisymmetric eigenvectors by integrating \eqref{eq:37} on the
interval $[0,\pi/2]$
\begin{align}
  \label{eq:f1_lambda}
  \lambda\int_0^{\pi/2} v^2w
  d\psi=-\int_0^{\pi/2}\left(v'^2+(k-1)\frac{\cos2\psi}{\sin^2\psi}v^2\right)\sin^{k-1}d\psi<0
\end{align}
As in the case of $f_0$ we can solve \eqref{eq:f0_L} analytically to
obtain
\begin{align}
  \label{eq:48}
  v_m^1(\psi)=\sin(\psi)C^{\left(\frac{1+k}{2}\right)}_m(\cos\psi), \\
  \label{eq:49}
  \lambda_m^1=2 + k (-1 + m) + m + m^2.
\end{align}
The only eigenvalue smaller than zero is as expected $\lambda_0^1=2-k$
and, by $C^{(a)}_0(x)=1$, the corresponding eigenvector is
$v_0^1=\sin\psi$.

\subsection{Stability of $f_2$ and $f_3$}
\label{sec:stability-f_2-f_3}

Although we cannot solve \eqref{eq:f0_L} analytically for $n\ge2$ we
know the number of negative eigenvalues from theorem
\ref{thm:harmonic-map-index} and their parity from \eqref{eq:50}.  We
utilize that knowledge to cut down the number of unstable directions
by restricting ourselves to the subspaces $H_+$ and $H_-$. This kills
all the antisymmetric modes for $f_2$ and the symmetric ones for $f_3$
reducing the index of both of them to $1$ in the respective
subspaces. We can thus write the perturbed solutions around $f_2$ and
$f_3$ as
\begin{align}
  \label{eq:53}
  f_+=f_2+A_0e^{-\lambda_2^0t}v_2^0+A_2e^{-\lambda_2^2t}v_2^2,\\
  f_-=f_3+B_1e^{-\lambda_3^1t}v_3^1+B_3e^{-\lambda_3^3t}v_3^3.
\end{align}
The first modes are the unstable ones while the latter are stable. We
shall use this prescription in ... .


%%% Local Variables:
%%% mode: latex
%%% TeX-master: "master"
%%% End:

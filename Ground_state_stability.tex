\section*{Stability of the ground states}

Given the solution $f_0=0$ we consider the small perturbation of the
form $f=f_0+e^{\lambda t} v$ with $v<<1$. The equation for $v$ is thus
of the form
\begin{align}\label{eq:f0_L}
  v''+(k-1)\cot\psi v'-(k-1)\frac{v}{\sin^2\psi}=\lambda v.
\end{align}
The critical exponents at $\psi=0$ are $1$ and $1-k$. The latter will
be rejected because the finitness of the energy condition yields that
the exponent $\alpha$ at $\psi=0$ has to fulfill
$\alpha>1-\frac{k}{2}$. Now when we know that $v$ and $v'$ are square
integrable we can multiply \eqref{eq:f0_L} by $\sin^{k-1}\psi v$ and
integrate on the interval $[0,\pi]$ to get
\begin{align}
  \lambda\int_0^\pi v^2w d\psi=-\int_0^\pi\left(v'^2+\frac{k-1}{\sin^2\psi}v^2\right)\sin^{k-1}d\psi<0.
\end{align}
All the terms under the integrals are positive for $v\ne0$ so
$\lambda<0$ and $f_0$ is linearly stable.\\

We use the similar reasoning for $f_1=\psi$ to get same critical
exponents and after integration on the interval $[0,\pi/2]$ we obtain
\begin{align}
  \label{eq:f1_lambda}
  \lambda\int_0^{\pi/2} v^2w d\psi=-\int_0^{\pi/2}\left(v'^2+(k-1)\frac{\cos2\psi}{\sin^2\psi}v^2\right)\sin^{k-1}d\psi<0
\end{align}
where we have used the antisymmetry of $v$ during the integration by
parts. This suggests that we could split the phase space into two
subspaces of symmetric and antisymmetric functions with respect to the
point symmetry around $(\psi,f)=(\pi/2,\pi/2)$. Such subspaces are
preserved by \eqref{eq:grad_flow}, we shall call them $H^+$ and $H^-$
respectively. Due to the stability properties of $f_0$ and $f_1$ we
will call them the ground states of $H^+$ and $H^-$. (TODO: $f_1$
minimizes energy for $H^-$).


%%% Local Variables:
%%% mode: latex
%%% TeX-master: "master"
%%% End:

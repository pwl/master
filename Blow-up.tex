\section*{Blow-up}

As we mentioned, apart from the first global attractor $f_0$ there
exists its image by the reflection symmetry $\bar{f}_0=\pi$ which is
also a global minimum of the energy. It turns out that before reaching
the second attractor the spatial derivative of the solution blows up
at the edges of the interval $[0,\pi]$. As the blow up is localized,
let's say at $\psi=0$, we can rewrite \eqref{eq:en_flow} up to first
corrections in $\psi$
\begin{align}
  \label{eq:2}
  \partial_t h=f''+\frac{(k-1)}{\psi}f'-\frac{k-1}{2}\frac{\sin2h}{\psi^2}.
\end{align}
This equation has the scaling symmetry of the form
$\psi\ra\lambda\psi$ and $t\ra\lambda^2t$ therefore we use the
self-similar ansatz of the form


% We shall now answer the question
% of how the convergence to this attractor is realized for initial data
% $g$ consistent with \eqref{eq:en_flow} for $\delta=0$
% \begin{align}
%   \label{eq:2}
%   f(\psi,0)=g(\psi)\\
%   g(0)=g(\pi)=0.
% \end{align}
% It is clear, that for $g$ close enough to $\pi$ in $L^2([0,\pi])$ the
% solution can have arbitrary small energy and thus should converge to
% $\pi$. However, the boundary conditions prevent the solution from
% converging to $\bar{f}_0$ uniformly,

% First of all, we notice that there exist continous functions $g$ of
% arbitrary small energy, yet close to $\pi$. Such functions are
% obtained by




%%% Local Variables:
%%% mode: latex
%%% TeX-master: "master"
%%% End:

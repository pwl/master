\section{Blow-up}
\label{sec:blow-up}

Although the theorem \ref{thm:Struwe} states that the blow up will
occur, it does not tell what form will it take. We shall now turn onto
the mechanism of the blow up for the generic initial data.\\

From now on assume that the blow up time $T$ is known a priori. The
blow up can be localized only on the edges of the interval
$[0,\pi]$. This can be argued by the fact that the energy
\eqref{eq:En_Sk} decreases so it is finite for all times. This in turn
implies that $f'$ has to be locally $L^2$ for $t<T$ inside the open
interval $(0,\pi)$, so the only places the $f'$ cannot be controlled
are the boundaries of $[0,\pi]$.\\

Because of the reflection symmetry $f(\psi)\ra f(\pi-\psi)$ of the
Euler-Legendre equations we can focus our analysis on one of the edges
solely, lets say $\psi=0$. Also as the possibly uncontrolled behaviour
is expected within the small neighbourhood of some given point on
$S^k$ the curvature of the domain of the map won't play any role so we
change the domain manifold to $\mathbb{R}^k$ being the space tangent
to $S^k$ at $\psi=0$. Therefore in this section we will analyse the
behaviour of the gradient flow for maps $\mathbb{R}^k\ra S^k$
satisfying the ansatz analogous to \eqref{eq:16} where the role of
$\psi$ is taken by the radial variable $r$ and $f$ is replaced by $h$.
For such map the energy is
\begin{align}
  \label{eq:51}
  E(h)=\frac{1}{2}\int_0^\infty \left(h'^2+(k-1)\frac{\sin^2 h}{r^2}\right)r^{k-1}dr,
\end{align}
while the equation for the gradient flow is
\begin{align}
  \label{eq:2}
  \partial_t
  h=h''+\frac{(k-1)}{r}h'-\frac{k-1}{2}\frac{\sin2h}{r^2}.
\end{align}
The first thing that distinguishes this equation from
\eqref{eq:en_flow} is that the above equation now possesses the
scaling symmetry
\begin{align}
  \label{eq:6}
  r&\ra\lambda r \\
  (T-t)&\ra\lambda^2(T-t)
\end{align}
where we recall $T$ as the blow up time. This motivates the use of the
self-similar variables of the form
\begin{align}
  \label{eq:3}
  y=\frac{y}{\sqrt{T-t}}\quad s=-\log(T-t)\quad H(s,y)=h(t,r),
\end{align}
to transform \eqref{eq:2} into
\begin{align}
  \label{eq:4}
  \partial_s
  H=H''+\left(\frac{(k-1)}{y}-\frac{y}{2}\right)H'-\frac{k-1}{2}\frac{\sin2H}{y^2}.
\end{align}
The given change of variables is tuned to the analysis of the
self-similar type of blow up which is the stationary points of the
above equation. If $H_0$ is such stationary point we have
\begin{align}
  \label{eq:5}
  H_0=H_0(y)=H_0\left(\frac{r}{\sqrt{T-t}}\right),
\end{align}
this $H_0$ is sometimes also called the self-similar anasatz. The
intuition behind such approach is that given an initial data
$h(0,r)=H_0(r/\sqrt{T})$, the solution to \eqref{eq:2} will
shrink with the rate $1/\sqrt{T-t}$ until it reaches some singular
state at $t=T$. Now, by the fact that the solution has to be regular
for any finite $r>0$ and any time $t$ including $t=T$ we have to
impose the condition on $H_0$
\begin{align}
  \label{eq:1}
  \forall{r>0}:\lim_{t\ra
    T}H_0\left(\frac{r}{\sqrt{T-t}}\right)=H_0(\infty)<\infty.
\end{align}
Such requirement, together with $0=f(t,0)=H(s,0)$, gives us the proper
boundary conditions for possible static solutions to \eqref{eq:4}. We
are therefore looking for the possible solutions to the ODE
\begin{align}
  \label{eq:7}
  H_0''+\left(\frac{(k-1)}{y}-\frac{y}{2}\right)H_0'-\frac{k-1}{2}\frac{\sin2H_0}{y^2},\\
  H_0(0)=0\quad H_0'(\infty)=0.
\end{align}
The existence of solutions to \eqref{eq:7} were proved in [] and
[]. Both authors have found the countable family of solutions
$\{H_n\}_{n\ge0}$ displaying some interesting properties (possibly
mentioned in the appendix), with the important one being the fact that
$n$-th solution has exactly $n$ extremas.\\

If we want the possible solution $H_n$ to appear as a final sate of
some class of the initial data broader than $H_n$ itself, we
additionally demand that $H_n$ is linearly stable. The stability
analysis leads to the eigenproblem of the operator
\begin{align}
  \label{eq:8}
  A_n u=u''+\left(\frac{(k-1)}{y}-\frac{y}{2}\right)u'-(k-1)\frac{\cos2H_n}{\sin
    y^2}u.
\end{align}
Using the identity \eqref{eq:strange_variation}, one can check, that
$A_n$ is bounded, self-adjoint and possesses an eigenvector $yH_n'(y)$
to the eigenvalue $-1$ (TODO:check the quantity). By the standard
results of the Sturm-Liouville theory we know that the eigenvector of
$A_n$ to the smallest eigenvalue shall have no zeroes at all, and the
eigenvectors to the successive eigenvalues ordered by magnitude shall
have one zero more then the previous ones. By this fact, and by the
number of zeroes of $H_n'$ we deduce that $A_n$ has $n$ eigenvalues
smaller that $-1$. This leaves us with the hypothesis that there are
no eigenvalues in the interval $(-1,0]$ which, by the time of writing
this thesis, can be verified only by numerical calculations.\\
The prescription $yH_n'(y)$ for the unstable mode, common for all the
static solutions, originates from our arbitrary choice of the blow up
time $T$, without the restrictions on the initial data. As it was
already mentioned, author is not aware of any explicit criterion for
the given initial data to blow up, not to even mention the time of the
blow up. Now as the coordinate system \eqref{eq:3} is fine tuned to
match the blow up time $T$, we shall ask what happens if the real blow
up time is a bit different, lets say $T'=T+\delta$. We shall have
\begin{align}
  \label{eq:9}
  H_0\left(\frac{r}{\sqrt{T+\delta-t}}\right)
  &=H_0\left(\frac{r}{\sqrt{T-t}}\right)-\delta\frac{1}{2}\frac{r}{(T-t)^{3/2}}H_0'\left(\frac{r}{\sqrt{T-t}}\right)+\mathcal{O}(\delta^2)\\
  &=H_0(y)-\delta\frac{1}{2} e^{-(-1)s} \underbrace{yH_0'(y)}_{\text{unstable
    mode}}+\mathcal{O}(\delta^2)
\end{align}
which means that the change of the blow up time by $\delta$ results in
the exponentially increasing perturbation in the coordinates
\eqref{eq:3}, thus the mode $yH_n'(y)$ arises from the coordinate
system not tuned to the blow up time.

Assuming that the last hypothesis (...) holds, the only stable blow up
profile is $H_0$. The more detailed analysis of the blow up scenario
and the possibility of continuation after the blow up will show up in
the, now in preparation, paper.


%%% Local Variables:
%%% mode: latex
%%% TeX-master: "master"
%%% End:

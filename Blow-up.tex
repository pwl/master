\section{Blow-up}
\label{sec:blow-up}

Although the theorem \ref{thm:Struwe} states that some kind of
singular behaviour will occur, it does not tell what form will it
take. We shall now turn onto the mechanism of the blow up for the
gradient flow of the maps $S^k\ra S^k$.\\

From now on we assume that the blow up time $T$ is known a priori. The
blow up can be localized only on the edges of the interval
$[0,\pi]$. (Jak to łatwo pokazać?).\\


% This can be argued by the fact that the energy
% \eqref{eq:20} is finite for $t\leT$ which is equivalent to $e(f)\in
% L^2([0,\pi],\sin^{k-1}\psi)$. However, at points

% inside the open interval $(0,\pi)$, so
% the only places the $f'$ cannot be controlled are the boundaries of $[0,\pi]$.\\

Because of the reflection symmetry $f(\psi)\ra f(\pi-\psi)$ of the
gradient flow equations we can focus our analysis on one of the edges,
lets say $\psi=0$. Also as the possibly uncontrolled behaviour is
expected only within some small neighbourhood of a pole of $S^k$ where
the curvature of the domain won't play any role so we change the
domain manifold to $\mathbb{R}^k$ being the space tangent to $S^k$ at
$\psi=0$. Therefore in this section we will analyse the behaviour of
the gradient flow for maps $\mathbb{R}^k\ra S^k$ satisfying the ansatz
analogous to \eqref{eq:17} but with the metric tensors of the domain
and the target manifolds given by
\begin{align}\label{eq:630}
  &\text{domain:}&\quad &ds^2=dr^2+r^2ds^2\big|_{S^{k-1}}\\
  &\text{target:}&\quad &dS^2=dh(r)^2+\sin^2h(r)dS^2\big|_{S^{k-1}}.
\end{align}
The role of $\psi$ is then taken by the radial variable $r$ and $f$ is
replaced by $h$. The Dirichlet energy of $h$ is
\begin{align}
  \label{eq:64}
  E(h)=\frac{1}{2}\int_0^\infty \left(h'^2+(k-1)\frac{\sin^2
      h}{r^2}\right)r^{k-1}dr,
\end{align}
from which the following tension field can be derived
\begin{align}
  \label{eq:65}
  \tau(h)=h''+\frac{(k-1)}{r}h'-\frac{k-1}{2}\frac{\sin2h}{r^2}.
\end{align}
The Euler-Lagrange equations for the stationary point of $E(h)$ are
\begin{align}
  \label{eq:63}
  \tau(h)=0,\\
  h(0)=0,\quad h(\infty)=m\pi.
\end{align}
The boundary condition at $\infty$ follows from the finiteness of
$E(h)$. As in the case of the maps $S^k\ra S^k$, \eqref{eq:64} has no
stable stationary points apart from $n\pi$. This can be shown using
the conformal transform of $r_A:\mathbb{R}^k\ra\mathbb{R}^k$ and the
rescaled function $h_A$ defined as follows
\begin{align}
  \label{eq:76}
  r_A(r)=e^A r,\quad h_A=h\circ r_A.
\end{align}
The energy of $h_A$ is
\begin{align}
  \label{eq:77}
  E(h_A)=e^{(2-k)A}E(h)
\end{align}
and therefore for any $h\ne n\pi$ the energy can be decreased by
rescaling the domain manifold. Also the tension field scales as
follows
\begin{align}
  \label{eq:78}
  \tau(h_A)_r=e^{2A}\tau(h)_{r_A}
\end{align}
where $\tau(h)_r$ is the tension field at the point $r$. The gradient
flow for $E(h)$ is defined as
\begin{align}
  \label{eq:66}
  \partial_t h=\tau(h).
\end{align}
The simple scaling property of the tension field motivates the ansatz
\begin{align}
  \label{eq:79}
  h(t,r)=H_0(e^{A(t)}r).
\end{align}
The idea behind such ansatz is that the solutions to \eqref{eq:66}
could consist of a constant profile sliding along the path generated
by the conformal transform. Substituting \eqref{eq:79} to
\eqref{eq:66} gives us
\begin{align}
  \label{eq:80}
  A'e^{-2A}yH_0'(y)=\tau(H_0)_y.
\end{align}
$H_0$ is the function of $y$ only, so the term involving $t$ has to be
constant and we get
\begin{align}
  \label{eq:86}
  A'e^{-2A}=C,\\
  \label{eq:81}
  e^{2A}=\frac{1}{\sqrt{2C(T-t)}},\\
  \label{eq:82}
  CyH_0(y)=\tau(H_0)_y.
\end{align}
The choice of the sign of $C$ corresponds to the choice between the
solutions which expand or shrink in time. The possibly singular
behaviour is displayed only by the function which shrinks in time
because if $H_0$ solves \eqref{eq:82}, its first spatial derivative at
$r=0$ is
\begin{align}
  \label{eq:83}
  \partial_r\left[ H_0(y)\right]_{r=0} =\frac{1}{\sqrt{2C(T-t)}}H_0'(0).
\end{align}
Therefore for $C>0$ as long as $H_0'(0)\ne0$ this quantity blows up in
time $T$. We can fix $C$ to $1/2$ by setting the argument of $H_0$ to
% The freedom to choose any $C>0$ is due to the rescaling of
% $y$ by some constant factor, there is however a canonical choice
% $C=1/2$ when the argument $y$ of $H_0$ becomes
\begin{align}
  \label{eq:84}
  y=\frac{r}{\sqrt{T-t}},\\
  \label{eq:85}
  \frac{1}{2}yH_0'(y)=\tau(H_0)_y.
\end{align}
Setting $h=H_0(y)$ with $y$ as in \eqref{eq:84} a priori to solving
\eqref{eq:86} is called the self-similar ansatz with $y$ being the
self-similar variable, as it is conserved by the self-similar symmetry
of \eqref{eq:66} given by
\begin{align}
  \label{eq:67}
  r&\ra\lambda r,\\(T-t)&\ra\lambda^2(T-t).
\end{align}
The complete set of self-similar variables consists of $y$ and the
logarithmic time $s$ defined by the relations
\begin{align}
  \label{eq:68}
  y=\frac{r}{\sqrt{T-t}},\quad s=-\log(T-t),\quad h(t,r)=H(s,y).
\end{align}

% To this point we haven't mentioned if the equation \eqref{eq:85} does
% have any solutions at all, there is however a paper by Fan
% \cite{Fan1999} in which the author proves the existence of the family
% of solutions to \eqref{eq:85} with the structure similar to that of
% harmonic maps.

In such variables \eqref{eq:66} takes the form
\begin{align}
  \label{eq:69}
  \partial_s
  H=H''+\left(\frac{(k-1)}{y}-\frac{y}{2}\right)H'-\frac{k-1}{2}\frac{\sin2H}{y^2}
\end{align}
The self-similar solution $H_0(y)$ is therefore a stationary point of
\eqref{eq:69}. By the fact that the solution has to be regular for any
$r>0$ and any time $t$ including $t=T$ we have to impose the condition
on $H_0$
\begin{align}
  \label{eq:71}
  \forall{r>0}:\lim_{t\ra T}
  \left|H_0\left(\frac{r}{\sqrt{T-t}}\right)\right|=\left|H_0(\infty)\right|<\infty.
\end{align}
Such requirement, together with $0=f(t,0)=H(s,0)$, gives us the proper
boundary conditions for possible static solutions to \eqref{eq:69}. We
are therefore looking for the solutions to the ODE
\begin{align}
  \label{eq:72}
  H_0''+\left(\frac{(k-1)}{y}-\frac{y}{2}\right)H_0'-\frac{k-1}{2}\frac{\sin2H_0}{y^2},\\
  H_0(0)=0\quad H_0'(\infty)=0.
\end{align}
Fan \cite{Fan1999} used ODE techniques to prove the existence of the
family of solutions to \eqref{eq:72} $H_n$, $n\ge0$, for $3\le k\le6$
with the structure similar to that of the harmonic maps between
spheres. Namely each $H_n$ has the nodal number $n$, $n-1$ extrema and
$H_n\in[0,\pi]$. The index of $H_n$ is however not known, but the
numerical evidence shows that $H_n$ is of index $n+1$. Comparing to
the harmonic maps $S^k\ra S^k$ each $H_n$ has one unstable direction
more than $f_n$, particularly $H_0$ is unstable. However, this
instability turns out to be the artifact of the self-similar variables
\eqref{eq:68}. As we recall, we have chosen the blow up time $T$ by
hand and the self-similar variables \eqref{eq:68} are fine tuned to
match this specific time. We shall ask what happens if the blow up
time for the given initial data is a bit different, lets say that it
is $T'=T+\delta$. We have
\begin{align}
  \label{eq:87}
  \begin{split}
    H_n\left(\frac{r}{\sqrt{T+\delta-t}}\right)
    &=H_n\left(\frac{r}{\sqrt{T-t}}\right)-\delta\frac{1}{2}\frac{r}{(T-t)^{3/2}}H_n'\left(\frac{r}{\sqrt{T-t}}\right)+\mathcal{O}(\delta^2)\\
    &=H_n(y)-\frac{\delta}{2} e^{s}
    yH_n'(y)+\mathcal{O}(\delta^2)
  \end{split}
\end{align}
which means that the change of the blow up time by $\delta$ results in
the exponentially increasing perturbation \eqref{eq:68} along
$yH_n'$. As this mode arises due to the ambiguity in the choice of the
self-similar variables it is also called the gauge mode, and does not
appear if $T$ is chosen appropriately ($\delta=0$). This means that by
the right choice of $T$ we can kill the gauge mode and claim the
following hypothesis.

\begin{conjecture}
  $H_n$ is of index $n$.
\end{conjecture}

In consequence, $H_0$ is the only linearly stable stationary point of
\eqref{eq:69} apart from $H=0$, and thus blow up is of self-similar
type and its profile is $H_0$.

% Assuming that the last hypothesis (...) holds, the only stable blow up
% profile is $H_0$. The more detailed analysis of the blow up scenario
% and the possibility of continuation after the blow up will show up in
% the, now in preparation, paper.


%%% Local Variables:
%%% mode: latex
%%% TeX-master: "master"
%%% End:

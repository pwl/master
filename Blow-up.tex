\section{Blow-up}
\label{sec:blow-up}

Although the theorem \ref{thm:Struwe} states that some kind of
singular behaviour will occur, it does not tell us what is the form of
blow-up. We shall now turn onto the mechanism of blow-up for the
gradient flow of the maps $S^k\ra S^k$.\\

From now on we assume that the blow-up time $T$ is known a priori. By
symmetry, the blow-up can be localized only on the edges of the
interval $[0,\pi]$.%(Jak to łatwo pokazać?).\\

Because of the reflection symmetry $f(\psi)\ra f(\pi-\psi)$ of the
gradient flow equations we can focus our analysis at one of the edges,
say $\psi=0$. Because the blow-up is localized within some small
neighbourhood of a pole of $S^k$, the curvature of the domain does not
play any role so we change the domain manifold to $\mathbb{R}^k$ being
the space tangent to $S^k$ at $\psi=0$. Therefore in this section we
will analyse the behaviour of the gradient flow for maps
$\mathbb{R}^k\ra S^k$ satisfying the ansatz analogous to \eqref{eq:18}
but with the metric tensors of the domain and the target manifolds
given by
\begin{align}\label{eq:71}
  &\text{domain:}&\quad &ds^2=dr^2+r^2ds^2\big|_{S^{k-1}}\\
  &\text{target:}&\quad &dS^2=dh(r)^2+\sin^2h(r)dS^2\big|_{S^{k-1}}.
\end{align}
The role of $\psi$ is then taken by the radial variable $r$ and $f$ is
replaced by $h$. The Dirichlet energy of $h$ is
\begin{align}
  \label{eq:72}
  E(h)=\frac{1}{2}\int_0^\infty \left(h'^2+(k-1)\frac{\sin^2
      h}{r^2}\right)r^{k-1}dr,
\end{align}
from which the following tension field can be derived
\begin{align}
  \label{eq:73}
  \tau(h)=h''+\frac{(k-1)}{r}h'-\frac{k-1}{2}\frac{\sin2h}{r^2}.
\end{align}
The Euler-Lagrange equations for the stationary point of $E(h)$ are
\begin{align}
  \label{eq:74}
  \tau(h)=0,\quad
  h(0)=0,\quad h(\infty)=m\pi.
\end{align}
The boundary condition at $\infty$ follows from the finiteness of
$E(h)$. As in the case of the maps $S^k\ra S^k$, \eqref{eq:72} has no
stable stationary points apart from $n\pi$. This can be shown using
the conformal transformation $r_A:\mathbb{R}^k\ra\mathbb{R}^k$ and the
rescaled function $h_A$ defined as follows
\begin{align}
  \label{eq:75}
  r_A(r)=e^A r,\quad h_A=h\circ r_A.
\end{align}
The energy of $h_A$ is
\begin{align}
  \label{eq:76}
  E(h_A)=e^{(2-k)A}E(h)
\end{align}
and therefore for any $h\ne n\pi$ the energy can be decreased by
rescaling the domain manifold. Also the tension field scales as
follows
\begin{align}
  \label{eq:77}
  \tau(h_A)_r=e^{2A}\tau(h)_{r_A}
\end{align}
where $\tau(h)_r$ is the tension field at the point $r$. The gradient
flow for $E(h)$ is defined as
\begin{align}
  \label{eq:78}
  \begin{split}
    \partial_t h=\tau(h),\quad
    h(0)=0,\quad h(\infty)=m\pi.
  \end{split}
\end{align}
The simple scaling property of the tension field motivates the ansatz
\begin{align}
  \label{eq:79}
  h(t,r)=H(e^{A(t)}r).
\end{align}
The idea behind such ansatz is that the solutions to \eqref{eq:78}
could consist of a constant profile sliding along the path generated
by the conformal transform. Substituting \eqref{eq:79} to
\eqref{eq:78} gives us
\begin{align}
  \label{eq:80}
  A'(t)e^{-2A(t)}yH'(y)=\tau(H)_y.
\end{align}
$H$ is the function of $y$ only, so the term involving $t$ has to be
constant and we get
\begin{align}
  \label{eq:81}
  &A'(t)e^{-2A(t)}=C,\\
  \label{eq:82}
  &e^{2A}=\frac{1}{\sqrt{2C(T-t)}},\\
  \label{eq:83}
  &CyH(y)=\tau(H)_y.
\end{align}
The choice of the sign of $C$ corresponds to the choice between the
solutions which expand or shrink in time. The possibly singular
behaviour is displayed only by the function which shrinks in time
because if $H$ solves \eqref{eq:83} its first spatial derivative at
$r=0$ is
\begin{align}
  \label{eq:84}
  \partial_r\left[ H(y)\right]_{r=0} =\frac{1}{\sqrt{2C(T-t)}}H'(0).
\end{align}
From \eqref{eq:83}, by the equation similar to \eqref{eq:46}, we know
that the only possible asymptotic behaviour of $H$ at $y=0$ fulfilling
$H(0)=0$ is $H\sim y$, therefore $H'(0)\ne0$. We conclude that for
$C>0$ the quantity \eqref{eq:84} blows up in time $T$. In the
literature this type of singularity is called the type I blow-up.
% (czy napisac cos wiecej na ten temat?).
We can fix $C$ to $1/2$ by setting the argument of $H$ to
\begin{align}
  \label{eq:85}
  y&=\frac{r}{\sqrt{T-t}},\\
  \label{eq:86}
  \frac{1}{2}yH'(y)&=\tau(H)_y.
\end{align}
By the fact that the solution to \eqref{eq:86} has to be regular for
any $r>0$ and any time $t$ including $t=T$ we have to impose the
condition on $H$
\begin{align}
  \label{eq:87}
  \forall{r>0}:\lim_{t\ra T}
  \left|H\left(\frac{r}{\sqrt{T-t}}\right)\right|=\left|H(\infty)\right|<\infty.
\end{align}
Such requirement, together with $0=f(t,0)=H(0)$, gives us the proper
boundary conditions for possible static solutions to \eqref{eq:91}. We
are therefore looking for the solutions to the ODE
\begin{align}
  \label{eq:88}
  \begin{split}
    &H''+\left(\frac{(k-1)}{y}-\frac{y}{2}\right)H'-\frac{k-1}{2}\frac{\sin2H}{y^2}=0,\\
    &H(0)=0,\quad H'(\infty)=0.
  \end{split}
\end{align}

There is also another approach to obtain \eqref{eq:88} called a
self-similar ansatz. It consist of setting $h=H(y)$ with $y$ as in
\eqref{eq:85} a priori to solving \eqref{eq:81}. It took its name from
the fact that such ansatz is invariant under the self-similar symmetry
of \eqref{eq:78} given by
\begin{align}
  \label{eq:89}
  r\ra\lambda r,\quad(T-t)\ra\lambda^2(T-t).
\end{align}
The complete set of self-similar variables consists of $y$ and the
logarithmic time $s$ defined by the relations
\begin{align}
  \label{eq:90}
  y=\frac{r}{\sqrt{T-t}},\quad s=-\log(T-t),\quad h(t,r)=H(s,y).
\end{align}
In such variables \eqref{eq:78} takes the form
\begin{align}
  \label{eq:91}
  \partial_s
  H=H''+\left(\frac{(k-1)}{y}-\frac{y}{2}\right)H'-\frac{k-1}{2}\frac{\sin2H}{y^2}.
\end{align}
The self-similar solution $H=H(y)$ is therefore a stationary point of
\eqref{eq:91}.\\

Fan \cite{Fan1999} used ODE techniques to prove the existence of the
family of solutions to \eqref{eq:88} $H_n$, $n\ge0$, for $3\le k\le6$
with the structure similar to that of the harmonic maps between
spheres. As in the case of $f_0$, there is the trivial solution
$H_0=0$. Then, for $n\ge1$ each $H_n$ has the nodal number $n$, $n-1$
extrema and $H_n\in[0,\pi]$. Also the index of $H_n$, albeit not known
analytically, can be found numerically to be $n$.

%%%
% For $n\ge1$ each $H_n$ possesses one unstable mode which turns out to
% be the artifact of the self-similar variables \eqref{eq:90}. As we
% recall, we have chosen the blow-up time $T$ a priori and the
% self-similar variables \eqref{eq:90} are fine tuned to match this
% specific time.
%%%

The original equation possesses a translational symmetry in the time
variable, this means that $H_n(r/\sqrt{T-t})$ is a solution to
\eqref{eq:78} for any choice of the blow-up time $T$. The ambiguity in
the choice of $T$ manifests itself as a gauge mode. To obtain the
explicit form of this mode we can ask what happens if the blow-up is
realized by $H_n$, but it occurs at time $T'$ which is a bit different
from $T$, say $T'=T+\delta$. In such situation we have
\begin{align}
  \label{eq:92}
  \begin{split}
    H_n\left(\frac{r}{\sqrt{T+\delta-t}}\right)
    &=H_n\left(\frac{r}{\sqrt{T-t}}\right)
    -\delta\frac{1}{2}\frac{r}{(T-t)^{3/2}}H_n'\left(\frac{r}{\sqrt{T-t}}\right)
    +\mathcal{O}(\delta^2)\\
    &=H_n(y)-\frac{\delta}{2} e^{s} yH_n'(y)+\mathcal{O}(\delta^2)
  \end{split}
\end{align}
which means that the change of the blow-up time by $\delta$ results in
the exponentially increasing perturbation along the mode $yH_n'$ with
the associated eigenvalue $-1$. This means, that among $n$ unstable
modes of $H_n$ there is one gauge mode and thus $H_1$ possesses no
unstable modes apart from the gauge one and the blow-up can be
realized by the profile of $H_1$.

% We can once again use Theorem \ref{thm:Struwe} applying it to the
% initial boundary problem \eqref{eq:78} to claim that if $m\ne0$ the
% blow-up will occur, and as $H_1$ is the only solution to \eqref{eq:91}
% which is stable apart from the gauge mode the blow up

% This means that,
% for the right choice of $T$ and $n\ge1$, $H_n$ has index $n-1$, and
% thus $H_1$ is linearly stable. Therefore, apart from the trivial
% solution $H_0$, there is one more final state of evolution as
% $s\ra\infty$ available for generic initial data, the state $H_1$ which
% is the profile of the self-similar blow-up solution.



%%% Local Variables:
%%% mode: latex
%%% TeX-master: "master"
%%% End:

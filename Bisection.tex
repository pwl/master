\section*{Bisection}

Now as we know, that for $k<k_{\text{crit}}$ equatorial map is of infinite
index and thus does not play any role in the dynamics of the
time-dependent equation we can proceed to the actual problem of
finding the solutions to the nonlinear equation of harmonic maps. Let
us first focus on the subspace $H_+$ (functions which are symmetric around
$\pi/2$). We assume
\begin{align}
  f(0,\psi)=f(0,\pi-\psi).
\end{align}
Our aim is to prepare the one parameter family of initial data i.e.
\begin{align}
  f(0,x)=g_A(x)=A\cdot g(x)
\end{align}
and observe how do the asymptotic states look like for different
choices of $A$. First of all, we take $g$ such that the flow for $A=1$
converges to $\bar{f}_0$. As mentioned, the convergence to $\bar{f}_0$
is preceded by the blow up of the first derivative at $\psi=0$.  On
the other hand, if $0<A<<1$ then $g_A$ is just a linear perturbation
of the ground state $f_0$, thus the flow will converge to
$f_0$.

% The sets of $A$
% for which $f$ converges to one of the ground states are open, which
% means that there is a closed set between them for which $f$ does not
% converge to any of the ground states. (TODO: formalize!). (TODO: what
% is an argument for such set to be one element only?).

% \begin{align}
%   \Gamma&=\{A\in\mathbb{R}_+|\lim_{t\ra\infty} f=f_0\}\\
%   \bar{\Gamma}&=\{A\in\mathbb{R}_+|\lim_{t\ra\infty} f=\bar{f}_0\}.
% \end{align}

With such choice of $g$ there will be only one $A^*$ such that for
$A<A^*$ solution tends to the ground state $f_0$ and for $A>A^*$ it
converges to $\bar{f}_0$ (TODO: how to show this?). For any initial
data the energy has to decrease along the flow and for $A=A^*$ it
cannot asymptotically decrease to $0$, so it has to be bounded from
below by some constant $E^*$
\begin{align}
  E(t)>E^*>0\quad E_t(t)<0.
\end{align}
Thus there is a limit
\begin{align}
  \lim_{t\ra\infty}E(t)=E^*\quad \lim_{t\ra\infty}E_t(t)=0
\end{align}
but, by \eqref{eq:en_t}, $E_t(t)\ra0$ can happen only if $f_t\ra0$ so
\begin{align}
  \lim_{t\ra\infty}f(t,\psi)=f^*(\psi)
\end{align}
which means, that $f^*$ is a symmetric solution to \eqref{eq:en_flow}
Apart From  $f_0$ As any change of $A^*$ in initial data $A^*\cdot g$
changes the attractor to one of the ground states, $f^*$ will have at
least one unstable mode due to the continuity of the solution
w.r.t. the perturbation in $A$. Moreover, as $f^*$ is nonconstant, it
has an additional unstable mode due to the Killing vector
$K=\sin(\psi)\frac{\partial}{\partial\psi}$. Those two unstable modes
do not coincide because $Kf^*$ is antisymmetric while $\partial_A f$
is symmetric (as the perturbated data is in $H^+$), so $f^*$ is of
index at least $2$. Given, we choose $g$ to be generic, we will obtain
the critical point $f_2:=f^*$ of index $2$.

%%% Local Variables:
%%% mode: latex
%%% TeX-master: "master"
%%% End:

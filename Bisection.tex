\section{Bisection}
\label{sec:bisection}

% We will now use the gradient flow \eqref{eq:en_flow} to implement the
% dynamical version of simple yet powerful technique called the mountain
% pass, which allows to prove the existence of the saddle point of the
% given functional. The intuition behind this technique is that given
% the path connecting two valleys which are separated by mountains,
% there has to be a point along that path from which if we have dropped
% a ball it won't fall into neither of the valleys, and thus it will has
% to roll down to some saddle point.\\

Although we already know, that $f_2$ and $f_3$ are saddle points of
$E(f)$ we shall demonstrate in this section how to obtain them using
only the gradient flow \eqref{eq:en_flow}.\\

% The implementation of this procedure to $H_+$ will look as follows.
Let us formally denote the basins of attraction of $0$ and $\pi$ as
$\Gamma(0)$ and $\Gamma(\pi)$ respectively. We say formally, because
the solutions to \eqref{eq:en_flow} will blow up (e.g. by Theorem
\ref{thm:Struwe}) before asymptotically converging to either of the
ground states. We start by choosing $g_+\in\Gamma(\pi)\cap H_+$ and
$g_+(0)=g_+(\pi)=0$. We now form a path $g_A\in H_+$ such that
\begin{align}
  \label{eq:93}
  g_A=A\cdot g_+,\quad A\in[0,1].
\end{align}
Obviously for $0\le A\ll1$, $g_A\in \Gamma(0)$, so the curve is hooked
up to both, $\Gamma(0)$ and $\Gamma(\pi)$. As $\Gamma(\pi)$ and
$\Gamma(0)$ are disjoint open sets, the curve leading from one of them
into the another has to contain the closed set which does not belong
to neither of the open sets. By definition, initial data from this set
cannot fall into any of those attractors. This means that there is at
least one $A^*$ such that the flow starting from $g_{A^*}$ is not
going to converge to the global energy minimum. Still, for any initial
data the energy has to decrease along the flow and for $A=A^*$ it
cannot asymptotically decrease to $0$, so its infimum $E^*$ obeys
\begin{align}\label{eq:94}
  E^*=\inf_{t\ge0} E(t)>0
\end{align}
and hence, if $g_{A^*}$ won't blow up, there limiting values are
\begin{align}\label{eq:95}
  \lim_{t\ra\infty}E(t)=E^*>0,\quad \lim_{t\ra\infty}\frac{dE}{dt}(t)=0
\end{align}
but, $dE/dt\ra0$ can happen only if $\partial_t f\ra0$ so
\begin{align}\label{eq:96}
  \lim_{t\ra\infty}f(t,\psi)=f^*(\psi)
\end{align}
which means, that $f^*$ is a symmetric solution to
\eqref{eq:f_psi_EL} with $E(f^*)=E^*$, different from the ground
states and, by construction, it is a saddle point in $H_+$. As $g_+$
was chosen to be generic, by such procedure we will obtain the saddle
point of the lowest index, namely $f^*=f_2$.\\

Analogous procedure can be applied to $H_-$ with $(\psi+g_-)\in
H_-\cap\Gamma(\pi-\psi)$, $g_-(0)=g_-(\pi)=0$ and
\begin{align}
  \label{eq:97}
  g_B=\psi+B\cdot g_-,\quad B\in[0,1]
\end{align}
to obtain $f_3$.\\



%%% Local Variables:
%%% mode: latex
%%% TeX-master: "master"
%%% End:
